
Cette étude a permis de démontrer que par une approche intégrative simple mettant en relation la
théorie des métapopulations et un modèle d'états et de transitions (STM), il est possible d'intégrer
implicitement ou explicitement des mécanismes écologiques tels que la dispersion, les interactions
biotiques et la démographie. Ces composantes écologiques sont essentielles afin de permettre
d'obtenir une prédiction réaliste de l'expansion de l'aire de distribution des espèces arborescentes
ou d'une communauté forestière. Plusieurs modèles de dynamique de la végétation (\textit{Dynamic
vegetation models}, DVMs) permettent aujourd'hui cette intégration par des approches plus
mécanistiques \citep{Snell2014a}. Ils intègrent par exemple la phénologie des espèces
\citep{Letters2001,Morin2008}, ou encore leurs capacités de dispersion \citep{Nobis2014,Iverson2004}
et peuvent ajouter une composante démographique \citep{Lischke2006a,Vanderwel2014}. Ces modèles
améliorent la qualité des prédictions, mais ne s'intéressent pas à la contribution de chaque
mécanisme écologique (dispersion, interaction biotique et démographie) afin de mieux comprendre
leurs effets sur la capacité d'expansion de l'aire de distribution d'une espèce ou d'une
communauté. En simulant la dynamique spatiale de l'écotone avec plusieurs variantes du STM, nous
avons réussi à isoler ces effets et dévoiler l'importance de la démographie et des interactions
biotiques comme contraintes à l'envahissement des espèces de la forêt tempérée en forêt boréale.
C'est de par ces contraintes que la forêt tempérée nordique ne sera pas en mesure de suivre sa
niche climatique d'ici la fin du siècle XXI$^e$ siècle.

\section*{Limites de l'approche de modélisation}

Une population possède trois façons de répondre aux  changements climatiques: elle se déplace,
s'acclimate (par plasticité phénotypique, court terme) ou encore s'adapte (par des changements dans
le génotype, long terme) \citep{Corlett2013}. La conceptualisation actuelle du modèle ne prend pas
en compte ce caractère adaptatif des espèces. Un réchauffement des conditions climatiques amène
certaines espèces à adapter leurs phénologies ou physiologies \citep{Saxe2001,Davis2001}.
L'adaptation des espèces boréales aurait pour conséquence, par exemple, d'accroitre la compétition
interspécifique et diminuer l'exclusion compétitive des espèces boréales par les espèces de milieux
tempérés. La présence de tels procédés adaptatifs conduit à modifier les taux de transitions entre
les différentes communautés boréales tempérées, limitant ainsi notre capacité à prédire la dynamique
de l'écotone dans le futur.

Les vitesses présentées dans l'étude de \citet{Davis1981} par l'analyse des registres polliniques se
caractérisent par une forte variabilité souvent associée à des évènements de dispersion longue
distance (Paradoxe de Reid's). Ce phénomène est rendu possible grâce aux organismes possédant une
forte mobilité telle que les oiseaux \citep{Clark1998}, mammifères, etc. Le scénario, intégrant une
dispersion limitée, ne permet pas de reproduire ces évènements stochastiques de dispersion. Ces
évènements peuvent pourtant permettre à une population de s'implanter et de persister, si les
conditions climatiques sont favorables, au-delà de leurs limites biogéographiques
\citep{Clark1998,Corlett2013}. L'établissement de population par ce mécanisme pourrait impliquer une
sous-estimation de la vitesse d'expansion du biome tempéré par le STM.  Sans évènements de dispersion
par longue distance, nous nous attendons à ce que la distance latitudinale augmente entre la
distribution potentielle de la communauté tempérée (sa niche climatique; $r>0$) et sa distribution
réalisée au cours des prochaines décennies. Ce déséquilibre entre les deux distributions a déjà été
observé pour plusieurs espèces européennes \citep{Svenning2004}.

\section*{Applicabilité du modèle}

En 2013, le Québec met en place la loi sur les forêts et se dote du cadre de l'aménagement
écosystémique. Cet aménagement vise à réduire les écarts entre la forêt naturelle et celle aménagée.
Il constitue un véritable défi pour les aménagistes puisqu'il signifie que la forêt doit retourner,
sur le long terme, vers sa composition et structure initiale. L'aménagement à la zone de transition
n'est pas évident considérant que cette zone est la plus susceptible à des changements de
composition.  La modification du couvert végétal peut conduire à une transformation rapide de la
composition des peuplements. Des changements abrupts de la composition par l'ouverture du couvert
ont déjà été observés à la zone de transition \cite{Dupuis2011,Boucher2006}. L'utilisation de ce
modèle dans le contexte d'aménagement écosystémique peut aider à anticiper la composition de la
régénération attendue au sein du peuplement après des interventions sylvicoles.

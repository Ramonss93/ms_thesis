[C’est dans cette section qu’est mise en évidence la portée de l’étude ainsi que les liens entre les articles ou autres textes et une ouverture sur les perspectives de recherche dans le domaine concerné. On y fait état des limites de la recherche et on y propose, le cas échéant, des pistes nouvelles pour de futures recherches ou des façons de développer de nouvelles applications. La conclusion ne doit pas présenter de nouveaux résultats ni de nouvelles interprétations. Elle doit être rédigée de manière à faire ressortir la cohérence de la démarche.]

Le modèle d'états et de transitions développé dans le cadre de cet étude a permis dévoiler que la forêt tempérée nodique ne sera pas en mesure de suivre sa niche climatique d'ici la fin du siècle. Cet incapacité est expliquable par la lente démographie des espèces tempérées et les fortes interactions biotiques avec les espèces boraéles. Ces deux composantes régissent la vitesse à laquelle les transitions des peuplements tempérée vers

Les modèles de dynamique de la végétation
(\textit{Dynamic vegetation models}, DVMs) permettent de combler certaines de ces limites par des
approches plus mécanistiques \citep{Snell2014a}. Ils intègrent par exemple la phénologie des espèces
\citep{Letters2001,Morin2008}, ou encore leurs capacités de dispersion \citep{Nobis2014,Iverson2004}
et peuvent ajouter une composante démographique \citep{Lischke2006a,Vanderwel2014}. Ces modèles
améliorent la qualité des prédictions, mais ne s'interressent pas à la contribution de chaque
mécanisme écologique (dispersion, interaction biotiques et démographie) afin de mieux comprendre
leurs effets sur la capacité d'expansion de l'aire de distribution d'une espèce ou d'une
communautée. Nous avons développés à travers cet étude une nouvelle méthologie permettant de tester

Limites du modèle:
Postulat
- Change in species assenblage
- Evolution of the niche, taille de la niche peut changer dans le temps

Contribution sur le plan écologique
- Comprendre et prédire la dynamique spatiale de l'écotone
- Importance pour les aménagistes
- Adaptation aux CC
- Aménager pour une cible en mouvement

Contribution sur le plan méthodologique
- Simplicité du modèle
- Modèle qui inclus implicitement la démographie (proba), les interaction biotiques (cells), distance de colonisation.
- Étendre la théorie des métapops à la biogéo

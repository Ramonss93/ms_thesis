
\chapter{La démographie, une contrainte à l'expansion de la forêt tempérée vers le Nord}

\section{Résumé en français du premier article}

De nombreuses espèces ne migrent pas assez vite pour suivre la rapidité des changements climatiques.  Les arbres sont bien connus pour éprouver de longs délais dans leurs réponses au climat parce qu'ils sont sessiles, longévive et dispose de faible capacité de dispersion. Les approches actuelles pour  prédire l'aire de répartition future des espèces, tels que les modèles d'enveloppe climatique, ne peuvent pas tenir compte de ces particularités propre aux écosystèmes forestiers car ils assument une dispersion infinie et une réponse instantanée aux changements climatiques. Nous proposons une nouvelle approche de modélisation basée sur la théorie des métapopulations pour tenir compte de cette capacité limitée de dispersion, des interactions biotiques et de la démographie propre à la forêt tempérée nordique du Nord-Est de l'Amérique du Nord. Notre objectif est d'évaluer si ce biome forestier sera en mesure de suivre sa niche climatique d'ici la fin de ce siècle. Nous avons effectué des simulations de l'écotone entre la forêt boréal et tempérée en utilisant un modèle d'états et de transitions (STM), dans lequel les communautés forestières sont classés dans 4 états: boréales, tempérées, mélangés et en régénération après une perturbation. Les transitions entre les états sont calibrés à partir des inventaires des parcelles permanentes présents aux États-Unis et au Canada. Les résutats des simulations du modèle indiquent que la forêt tempérée se déplacera seulement 14 $ \ pm $ 2,0 km d'ici la fin de ce siècle. contrairement aux prédictions des modèles de distribution d'espèces standards que la forêt tempérée se déplacera vers le nord par 238,79 $ \ pm 34,24 $ km. Nous constatons également que la plupart des transitions attendues sera la conversion du mélange à des peuplements purs tempérées. Une comparaison avec un scénario de dispersion infinie révèle que les interactions biotiques et la dynamique de remplacement des peuplements sont les facteurs les plus importants qui limitent le taux des arbres forestiers de migration. Nous concluons que la forêt tempérée a une faible résilience au changement climatique en raison de leur faible démographie et les interactions compétitives avec des arbres résidents.

De nombreuses espèces ne migrent pas assez vite pour suivre la rapidité des changements climatiques. Les arbres sont bien connus pour éprouver de longs délais dans leurs réponses au climat parce qu'ils sont sessiles, longévive et dispose de faible capacité de dispersion. Les approches actuelles pour prédire l'aire de répartition future des espèces ne peuvent pas tenir compte de ces particularités propre aux écosystèmes forestiers car ils assument une dispersion infinie et une réponse instantanée aux changements climatiques. À travers cette étude, nous proposons une nouvelle approche de modélisation basée sur la théorie des métapopulations pour tenir compte de la dispersion limitée, des interactions biotiques et de la démographie de la forêt tempérée. Notre objectif est d'évaluer si la forêt tempérée d'Amérique du Nord-Est sera en mesure de suivre son optimum climatique d'ici la fin du siècle. Les transitions entre les états sont calibrés à partir de plusieurs enquêtes sur les parcelles forestières à long terme des États United- et au Canada. Nous constatons que, même si les modèles de distribution des espèces classiques prédisaient un déplacement vers le nord de la distribution de la forêt tempérée de 328 $\pm$ 28,4 kms, la forêt tempérée sera à peine bouger 14 $\pm $ 2,0 km dans la forêt boréale à la fin de ce siècle. Nous constatons également que la plupart des transitions attendues sera la conversion du mélange à des peuplements purs tempérées. Une comparaison avec un scénario de dispersion infinie révèle que les interactions biotiques et la dynamique de remplacement des peuplements sont les facteurs les plus signifcatifs limitant le taux des arbres forestiers de migration. Nous concluons que la forêt tempérée a une faible résilience au changement climatique en raison de leur faible démographie et les interactions compétitives avec des arbres résidents.

En plus d’y consigner les grandes lignes de l’article, cette section sert de lieu pour préciser le contexte du projet. On y mentionne le nom de la revue où l’article a été soumis ainsi que le stade d’avancement de l’évaluation par les pairs. Un sommaire de la contribution de chacun des auteurs doit aussi être présenté. Voir l’exemple suivant :

Ce premier article, intitulé \enquote{\textit{Combining Apparent Motion and Perspective as Visual Cues for Content-based Camera Motion Indexing}}, fut corédigé par moi-même ainsi que par le professeur Frédéric Deschênes et ma collègue Joanie Pan. Il fut accepté pour publication dans sa version finale en 2009 par les éditeurs de la revue \textit{Pattern Recognition}. En tant que premier auteur, ma contribution à ce travail fut l’essentiel de la recherche sur l’état de l’art, le développement de la méthode, l’exécution des tests de performance et la rédaction de l’article. Le professeur Frédéric Deschênes, second auteur, a fourni l’idée originale. Il a aidé à la recherche sur l’état de l’art, au développement de la méthode ainsi qu’à la révision de l’article. Joanie Pan, troisième auteure, a contribué à la recherche sur l’état de l’art ainsi qu’à l’exécution des tests de performance. Une version abrégée de cet article a été présentée à la conférence \textit{Canadian Conférence on Computer and Robot Vision} à Washington D.C. (É.-U.) à l’automne 2008.]

\newpage

\section{Slow demography constrains the North-Eastern Temperate Forest expansion under Climate Change}

\textbf{AUTHORSHIP}


\chapter{La démographie, une contrainte à l'expansion de la forêt tempérée vers le Nord}

\section{Résumé en français du premier article}

De nombreuses espèces ne migrent pas assez vite pour suivre la rapidité des changements climatiques.
Les arbres sont bien connus pour éprouver de longs délais dans leurs réponses au climat parce qu'ils sont sessiles, possèdent une forte longévité et disposent de faible capacité de dispersion. Les approches actuelles pour prédire l'aire de répartition future des espèces, telles que les modèles d'enveloppe climatique, ne peuvent pas tenir compte de ces particularités propres aux écosystèmes forestiers, car ils assument une dispersion infinie et une réponse instantanée aux changements climatiques. Nous proposons une nouvelle approche de modélisation basée sur la théorie des métapopulations pour tenir compte de cette capacité limitée de dispersion, des interactions biotiques et de la démographie propre à la forêt tempérée nordique du nord-est de l'Amérique du Nord. Notre objectif est d'évaluer si ce biome forestier sera en mesure de suivre sa niche climatique d'ici la fin de ce siècle. Nous avons effectué des simulations de l'écotone entre la forêt boréale et tempérée en utilisant un modèle d'états et de transitions (STM), dans lequel les communautés forestières sont classées dans 4 états: boréales, tempérées, mélangées et en régénération après une perturbation. Les transitions entre les états sont calibrées à partir des inventaires des parcelles permanentes présents aux États-Unis et au Canada. Les résultats des simulations du modèle indiquent que la forêt tempérée se déplacera seulement de 14 $\pm$ 2,0 km alors qu'un modèle de distribution d'espèces standard prédit un déplacement de 238,79 $\pm$ 34,24 km. Les simulations de l'écotone forestier mettent également en évidence que la majorité des transitions attendues seront une conversion des peuplements mixtes vers des peuplements purement décidus. L'utilisation du modèle avec un scénario de dispersion infinie révèle que les interactions biotiques et la démographie sont les facteurs les plus importants qui limitent la capacité d'expansion du biome de la forêt tempérée. En conclusion, la forêt tempérée possède une faible résilience au changement climatique en raison de sa lente démographie et des fortes interactions compétitives avec les espèces boréales.

Ce premier article, intitulé \enquote{\textit{Slow demography constrains the North-Eastern Temperate
Forest expansion under Climate Change}}, fut corédigé par moi-même ainsi que mon Professeur
Dominique Gravel et mes deux cosuperviseurs, Matthew Talutto (Ph.D) et Isabelle Boulangeat (Ph.D).
L'article présenté sera soumis à \enquote{\textit{Global Change Biologie}} pour évaluation par mes
pairs à la fin de l'été 2016. Cet article constitue l'un des volets du projet stratégique QUICC-FOR,
financé par le CRSNG, qui vise à cartographier et quantifier les conséquences des changements
climatiques sur les forêts de l'Est de l'Amérique du Nord. Ma contribution en tant qu'auteur se
résume en cinq points: (i) effectuer un état de la littérature; (ii) conceptualiser le modèle et
l’implémenter grâce au langage de programmation C; (iii) créer une base de données nécessaire à la
calibration et la validation du modèle; (iv) effectuer le post-traitement et l'analyse des
simulations; (v) rédiger l'article. Dominique Gravel est à l'origine de l'idée du projet et a aidé à
la conceptualisation, la validation du modèle et la révision du manuscrit. Matthew Talluto est
responsable de la calibration bayésienne avec la méthode MCMC (\textit{Monte Carlo Markov Chain}).
Il a également contribué à l'implémentation du modèle en C ainsi qu'à la révision du manuscrit.
Isabelle Boulangeat est responsable de l'estimation des paramètres par maximum de vraisemblance
nécessaire à l'initialisation du MCMC. Elle a également contribué à la révision du manuscrit.
L'ensemble de mon équipe d'encadrement a fourni une assistance technique inestimable dans la plupart
des étapes scientifiques nécessaires à l'obtention de ces résultats.

Les résultats présentés ici reposent sur les paramètres issus la calibration par maximum de vraisemblance. De nouvelles simulations vont être amorcées en utilisant la distribution postérieure du MCMC afin d'obtenir une meilleure estimation de l'incertitude dans les projections. Les résultats de cet article ont été présentés à deux congrès internationaux et deux congrès provinciaux sous forme d'une affiche et d'une conférence. La conférence s'intitulait \enquote{\textit{Difficult migration of temperate tree species in boreal forest under climate change?}}, présenté au 9$^e$ colloque du Centre d'étude de la Forêt (CEF) en avril 2015 et au 7$^e$ congrès eCANUSA sur les sciences forestières en octobre 2014. Enfin, l'affiche portait le titre \enquote{\textit{A state transition model to investigate what constrains the northward migration of the temperate forest}} et a fait l'objet d'une présentation au colloque du Centre des sciences de la biodiversité du Québec (CSBQ) en octobre 2015 et au 9$^e$ congrès IALE en écologie du paysage en juillet 2015. L'ensemble de ces travaux peuvent être téléchargés à partir de mon site internet personnel (\url{http://steveviss.github.io/paper/}). En parallèle à cette maîtrise, je me suis impliqué à titre de coauteur dans l'un des chapitres de la thèse de Christian Marchese (UQAR) portant sur la phénologie du phytoplancton dans la région arctique de la mer de Baffin. Cette collaboration a aboutie à un article intitulé \enquote{\textit{Changes in phytoplankton bloom phenology in the NOW polynya region: a response to changing environmental conditions}}, soumis en mars 2016 dans la revue \textit{Polar Biology}.




\newpage

\section{Slow demography constrains the North-Eastern Temperate Forest expansion under Climate Change}

\textbf{AUTHORSHIP}

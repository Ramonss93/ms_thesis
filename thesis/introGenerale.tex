% L’introduction générale doit présenter la problématique des travaux de recherche, les principales dimensions méthodologiques, s'il y a lieu, et les résultats obtenus directement en rapport avec la contribution de celui qui produit la thèse ou le mémoire. Il faut bien montrer comment se situe la contribution originale de l'auteur (ou des auteurs) par rapport aux travaux cités dans la liste des références bibliographiques. Les sous-titres de l’introduction générale ne sont pas numérotés.

\section*{Mise en contexte}
\addcontentsline{toc}{section}{\protect\numberline{}Mise en Contexte}

Depuis l'ère industrielle, la forêt du Québec méridionale est en constante évolution (Arsenault); le
paysage forestier tel que nous le connaissons aujourd'hui pourrait connaître de profondes
modifications d'ici la fin du XXI$^e$ siècle. Ce paysage est occupé en grande majorité par la forêt
tempérée qui couvre une superficie de 209 700 km$^2$ \citep{Boulay2015}. Cette forêt peut ainsi être
désignée comme la forêt habitée du Québec considérant qu'elle se retrouve dans la zone la plus
densément peuplée du Québec \citep{Doyon2009}. On y retrouve une multitude et une diversité
d'activités socio-économiques tels que le tourisme, la chasse et l'acériculture et le prélèvement
sylvicole. Au Québec, l'industrie forestière et l'acériculture génèrent 21,6 et 321,7 millions de
dollars respectivement \citep{Boulay2015}. La prospérité de ces activités repose sur l'intégrité
écologique de ce biome forestier régionale. Sa gestion est donc primordiale, mais constitue un
véritable défi de par la diversité des acteurs socio-économique, certains enjeux écologiques et les
attentes de la société. Ce sont ces mêmes attentes qui ont contribué à l'adoption en 2013 de la loi sur l'aménagement durable du territoire forestier visant à maintenir la diversité biologique et la viabilité de ces écosystème \citep{Boulay2015}.

Depuis maintenant plusieurs années, la forêt tempérée est confrontée à de nombreux enjeux
écologiques tels que les problématiques d'enfeuillement, la raréfaction de certaines essences ou
envahissement par d’autres, la simplification des structures internes des peuplements
\citep{Varady-Szabo2008}. Aujourd'hui, la forêt tempérée nordique doit faire face à une nouvelle
problématique qui est celle des changements climatiques. Plusieurs enjeux écologiques majeurs
découlent de cet problématique pour les aménagistes: (1) des modifications dans la composition de la
régénération post-perturbation; (2) une modulation de la productivité forestière chez certaines
espèces; (3) une modification du régime de perturbation (p.ex. épidémies, verglas, chablis); puis
enfin (4) des changements dans la répartition des espèces. Ce mémoire porte sur ce quatrième volet
en s'interressant à la biogéographie et la dynamique de la communauté de la forêt tempérée nordique
dans ce contexte de changements climatiques (c.a.d. son écotone).

\section*{Cadre conceptuel}
\addcontentsline{toc}{section}{\protect\numberline{}Cadre conceptuel}

Le climat exerce un contrôle dominant sur la distribution des espèces à une échelle régionnale
\citep{Pearson2003b}. L'étude des registres polliniques démontrent que les fluctuations climatiques
de la période du Quaternaire ont engendré des contractions et expansions dans l'aire de distribution
des espèces \citep{Davis2001}. Aujourd'hui, l'effet des changements climatiques est déjà notable sur
la diversité végétale \citep{Walther2002a,Parmesan2006}. Considérant l'augmentation de température
de 4 à 7$^\circ$C \citep[Scénario RCP 8.5]{Climatique2015} attendu pour le Québec méridionale,
plusieurs études prédisent que des espèces de la forêt tempérée (\textit{Fagus grandifolia},
\textit{Betula alleghaniensis}, \textit{Acer saccharum}, \textit{Quercus rubra}) vont étendre leur
distribution vers le nord d'ici la fin du siècle \citep{Iverson2002,Sciences2014}. Cependant, ces
changements dans la composition végétale pourrait s'avérer difficile considérant que les
microconditions environnementales que l’on retrouve sous les espèces boréales, sont différentes de
celles présentes en forêt tempérée \citep{Barras1998,Caspersen2005}. Ainsi, même si les conditions
climatiques à l'échelle de la région sont favorables à l'établissement des espèces tempérée, les
microconditions édaphiques et climatiques particulières à la forêt boréale, pourraient nuire à
l'établissement des espèces tempérée \citep{DeFrenne2013,Lafleur2010}. Si c’est le cas, il est
possible que la forêt tempérée nordique ne parvienne pas à s'installer en forêt boréale à la suite
d’un réchauffement climatique.

L'une des approches classiques en biogéographie pour prédire les changements d'aire de répartition
d'une espèce réside dans l'utilisation de modèles d'enveloppe bioclimatique \citep{Pearson2003b}.
Ces modèles corrélatifs mettent en relation la présence et l'abscence d'une espèce avec le climat
\citep{Guisan2005a}. Ils présentent plusieurs limites puisqu'ils se basent sur la niche fondamentale
et ne tiennent pas compte des facteurs biotiques (p. ex. capacité et taux de dispersion, compétition
inter-spécifique) influençant l'aire de répartition des espèces \citep{Guisan2005a,Pearson2003b}.
Ils assument également que la végétation en place est en équilibre avec le climat et qu'elle répond
de manière instantanée à des changements environnementaux \citep{Austin2002}. Ils sont donc
statiques de part l'exclusion des processus de dispersion, succession ou encore de perturbations
tels que le feu ou le broutage \citep{Austin2002,Guisan2005a}. Les modèele de dynamique de la
végétation (\textit{Dynamic vegetation models}, DVMs) permettent de combler certaines de ces lacunes
par des approches plus mécanistiques \citep{Snell2014a}. Ils intègrent la phénologie des espèces
\citep{Letters2001,Morin2008}, ou encore leurs capacités de dispersion \citep{Nobis2014,Iverson2004}
et enfin peuvent ajouter une composante démographique \citep{Lischke2006a,Vanderwel2014}. Ces
modèles tentent d'améliorer la qualité des prédictions des aires de répartition futures des espèces
mais ne s'interresse pas intégrer l'ensemble des mécanismes écologiques (dispersion, interaction
biotiques et démographie) pour comprendre leurs conséquences sur la capacité de migration des
espèces. 

L'expansion de la forêt tempérée nordique à la marge supérieure de son aire de répartition doit être
vu comme un long processus écologique. Ce processus repose sur la survie des arbres matures
(démographie), la production et la dissémination des graines (dispersion), et enfin le succès de
germination, la survie et la croissance des semis dans les habitats adjacents à sa distribution
(établissement) \citep{Schurr2012,Travis2013}. Ce processus est donc lent considérant la faible
capacité des espèces tempérée à se disperser, leurs fortes longévités, et enfin leurs faibles
faibles taux de croissance \citep{Renwick2014,Vanderwel2014}.

Ces évènement peuvent être reporter à une suite d'évènements de colonisation et d'extinction à une échelle locale. Si le taux de colonization excède le taux d'extinction alors l'éspèce sera en mesure de s'établir et persister au sein du peuplement.  

% 			- S'adapter ou changer de place
% 			- Migration chez les arbres se traduit par une suite d'évènement de colonisation et d'extinction pour
% 			Pour comprendre les mécanismes déclencheurs d'une migration potentielle. On ne parle pas ici d'adaptation mais de résistance et inertie du système aux changements.
% 			- Postulat Majeur: évolution - taille de la niche change pas dans le temps
% 			- Towards the edge of the distribution, growth and establishment is reduced, while growth efficiency-related mortality increases. (Thullier 2008).
%

% \textbf{3e.}  Concilier la théorie de la niche avec celle des métapops: une avenue pour régler ce problème ?
% 	Nouvelle génération de modèle plus mécanistiques: Démographie + dispersion
% - Théorie de la niche
% - Théorie métapop
% - Lien le concept la niche d'établissement, peristence ().
% - Reprendre le schema de Holt et replacer les équation de levins dedans
%
% - Limite des arbres à la migration
%
% - Outils disponible
%
% Pour comprendre et prédire, ces changements
% . Dessiné par des contraintes physiologiques à échelle régionnale mais à plus fine échelle des contraintes biotiques et abiotiques
% - Est-ce que la forêt tempérée suivra le rythme ?
% Depuis maintenant 25 ans, l'écotone entre la forêt boréal et la forêt tempérée nordique fait l'objet d'une attention particulière \citep{Goldblum2010}. Cette
% Rapidité des changements. Connaitre la vélocité de ses changements est primodiale pour améliorer notre capacité à adapter nos pratiques sylvicole.
%
% Migration pour des espèces végétale, en quoi ca consiste ?
%
% Outils disponibles pour étudier la migration
%
% Pour comprendre ces changements dans la distribution,
% l'une des approches classiques centrée sur l'espèce consiste à produire des modèles dit corrélatif.
% se rapproche de la niche fondamentale
% - Thuillier + Guisan (2005)
% Puisan (2000)
%
% Pour prédire des changemens d'aire de répartition, deux grandes familles de modèles éxistent:
% 	- SDM (BIOMOD2, enveloppe bioclimatique)
% 	- DVM tente de corriger par des approches plus mécanistiques en tenant compte de la phénologie des espèces (PhenoFit, Morin), se capacité à se disperser (TreeMig, Liscke, SHIFT), démographie (CAIN, Vanderwhel).
% 	- En couplant, SDM avec modèles de migration CAIN, SHIFT.
% - Certains postulats ne sont pas approprié lorsque l'on tente de modéliser le réponse des arbres.
% - Conclusion du paragraphe, nécéssité de se dirger vers une nouvelle générationde modèle
%
% Les changements d'aire de répartition sont aujourd'hui documenté par des approches corrélatives. Ces outils sont limités pour prédire des changements de répartition pour des espèces possédant une forte longévité, une capacité de dispersion limité et oû la compétition est
%
% Les outils disponibles et limites pour étudier ce phénomène. Finir avec l'importance d'intégrer la dispersion et la démographie
%
%
% 	Limites:
% 	- Réponse instantanée
% 	- Importance des processus démographique et de dispersion (Holt,2005)
% 			- Conduit vers
%
% \textbf{1er}. RECENTRER SUR LE PROBLÈME ÉCOLOGIQUES:  À quoi est ce que l'on s'interresse lorsque l'on parle de changement d'aire de répartition chez les espèces arborescentes? Définir le concept de migration comme un long processus qui repose sur la la démographie (la survie et l'atteinte de la maturité), la dispersion (production et dissémination des graine), l'établissement (le succès de germination, la survie et la croissance des semis) (Voir Travis)
% 			- S'adapter ou changer de place
% 			- Migration chez les arbres se traduit par une suite d'évènement de colonisation et d'extinction pour
% 			Pour comprendre les mécanismes déclencheurs d'une migration potentielle. On ne parle pas ici d'adaptation mais de résistance et inertie du système aux changements.
% 			- Postulat Majeur: évolution - taille de la niche change pas dans le temps
% 			- Towards the edge of the distribution, growth and establishment is reduced, while growth efficiency-related mortality increases. (Thullier 2008).
%

%
% \textbf{4e.}  Description du système et transition vers le cadre méthodologique
%
% 
%
% - Les approches de modélisation
% - Étude des taux de migration
% - Objectif
%
% \section*{\uppercase{OBJECTIFS DE L'ÉTUDE}}
% - À partir des données disponibles
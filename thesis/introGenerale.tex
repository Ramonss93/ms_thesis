% L’introduction générale doit présenter la problématique des travaux de recherche, les principales dimensions méthodologiques, s'il y a lieu, et les résultats obtenus directement en rapport avec la contribution de celui qui produit la thèse ou le mémoire. Il faut bien montrer comment se situe la contribution originale de l'auteur (ou des auteurs) par rapport aux travaux cités dans la liste des références bibliographiques. Les sous-titres de l’introduction générale ne sont pas numérotés.

\section*{\uppercase{Mise en contexte}}
\addcontentsline{toc}{section}{\protect\numberline{}MISE EN CONTEXTE}

Depuis l'ère industrielle, la forêt du Québec méridionale est en constante évolution (Arsenault); le
paysage forestier tel que nous le connaissons aujourd'hui pourrait connaître de profondes
modifications d'ici la fin du XXI$^e$ siècle. Ce paysage est occupé en grande majorité par la forêt
tempérée qui couvre une superficie de 209 700 km$^2$ (MFFP, 2015). Cette forêt peut ainsi être
désignée comme la forêt habitée du Québec considérant qu'elle se retrouve dans la zone la plus
densément peuplée du Québec (Doyon). On y retrouve une multitude et une diversité d'activités
socio-économiques tels que le tourisme, la chasse et l'acériculture et le prélèvement sylvicole. Au
Québec, l'industrie forestière et l'acériculture génèrent XX et XX de dollars respectivement pour un
total de XX millions d'emplois en 2015 (MFFP, Stats). La prospérité de ces activités repose sur
l'intégrité écologique de ce biome forestier régionale. Sa gestion est donc primordiale, mais
constitue un véritable défi de par la diversité des acteurs socio-économique, certains enjeux
écologiques et les attentes de la société. Ce sont ces mêmes attentes qui ont contribué à l'adoption
en XXXX d'un plan d'aménagement écosystémique visant à maintenir la diversité biologique et la
viabilité de cet écosystème (MFFP, ).

Depuis maintenant plusieurs années, la forêt tempérée est confrontée à de nombreux enjeux
écologiques tels que les problématiques d'enfeuillement, la raréfaction de certaines essences ou
envahissement par d’autres, la simplification des structures internes des peuplements
(Varady-Szabo). Aujourd'hui, la forêt tempérée nordique doit faire face à une nouvelle problématique
qui est celle des changements climatiques. Plusieurs enjeux écologiques majeurs découlent de cet
problématique pour les aménagistes: (1) des modifications dans la composition de la régénération
post-perturbation; (2) une modulation de la productivité forestière chez certaines espèces; (3) une
modification du régime de perturbation (p.ex. épidémies, verglas, chablis); puis enfin (4) des
changements dans la répartition des espèces. Ce mémoire porte sur ce quatrième volet en s'interressant à la biogéographie et la dynamique de la communauté de la forêt tempérée
nordique dans ce contexte de changements climatiques (c.a.d. son écotone).

\section*{\uppercase{CADRE CONCEPTUEL}}
\addcontentsline{toc}{section}{\protect\numberline{}ÉTAT DES CONNAISSANCES} 

La biosphère a déjà connu plusieurs épisodes de changements climatiques. L'étude des registres
polliniques démontrent que ces fluctuations climatiques passées ont engendré des contractions et
expansions dans l'aire de distribution des espèces (e.g., Davis and Shaw 2001). Aujourd'hui, l'effet
des changements climatiques de ce siècle est déjà observable sur la diversité végétale et animale (Parmesan and Yohe 2003; Walther et
al. 2002). Considérant l'ampleur et la vélocité des changements climatiques prédis pour le XXIe
siècle (IPCC, 2015), la forêt tempérée nordique sera-t-elle en mesure de déplacer son aire de
distribution assez rapidement pour suivre son enveloppe climatique?

% 
% - Limite des arbres à la migration
% 
% - Outils disponible
% 
% Pour comprendre et prédire, ces changements 
% . Dessiné par des contraintes physiologiques à échelle régionnale mais à plus fine échelle des contraintes biotiques et abiotiques
% - Est-ce que la forêt tempérée suivra le rythme ?
% Depuis maintenant 25 ans, l'écotone entre la forêt boréal et la forêt tempérée nordique fait l'objet d'une attention particulière \citep{Goldblum2010}. Cette  
% Rapidité des changements. Connaitre la vélocité de ses changements est primodiale pour améliorer notre capacité à adapter nos pratiques sylvicole. 
% 
% Migration pour des espèces végétale, en quoi ca consiste ?
% 
% Outils disponibles pour étudier la migration 
% 
% Pour comprendre ces changements dans la distribution, 
% l'une des approches classiques centrée sur l'espèce consiste à produire des modèles dit corrélatif. 
% se rapproche de la niche fondamentale
% - Thuillier + Guisan (2005)
% Puisan (2000)
% 
% Pour prédire des changemens d'aire de répartition, deux grandes familles de modèles éxistent:
% 	- SDM (BIOMOD2, enveloppe bioclimatique)
% 	- DVM (TreeMig, PhenoFit) tente de corriger par des approches plus mécanistiques en tenant compte de la phénologie des espèces (PhenoFit, Morin) et se capacité à se disperser (TreeMig, Liscke).
% 	- En couplant, SDM avec modèles de migration CAIN, SHIFT.
% - Certains postulats ne sont pas approprié lorsque l'on tente de modéliser le réponse des arbres. 
% - Conclusion du paragraphe, nécéssité de se dirger vers une nouvelle générationde modèle 
% 
% Les changements d'aire de répartition sont aujourd'hui documenté par des approches corrélatives. Ces outils sont limités pour prédire des changements de répartition pour des espèces possédant une forte longévité, une capacité de dispersion limité et oû la compétition est
% 
% Les outils disponibles et limites pour étudier ce phénomène. Finir avec l'importance d'intégrer la dispersion et la démographie
% 
% Les modèles d’enveloppe bioclimatique possèdent plusieurs limites. Ces outils se basent sur la niche fondamentale d’une espèce et ne tiennent donc pas compte des facteurs biotiques (p. ex. capacité et taux de dispersion, compétition inter-spécifique) influençant l’aire de répartition de cette dernière [Guisan2005, Pearson2003]. Ces facteurs jouent pourtant un rôle prépondérant dans la dynamique d’un écosystème [Guisan2005, Araujo2007, Pearson2003]. De plus, ces modèles statistiques assument que la végétation a atteint partiellement ou complètement son point d’équilibre avec le climat [Austin2002]. Ils sont donc statiques et ne contiennent aucune composante dynamique (p. ex. les processus de dispersion, succession ou encore différents agents de perturbations comme le feu ou le broutage) [Guisan2005, Austin2002].
% 	
% 	Limites:
% 	- Réponse instantanée 
% 	- Importance des processus démographique et de dispersion (Holt,2005)
% 			- Conduit vers 
% 			
% \textbf{1er}. RECENTRER SUR LE PROBLÈME ÉCOLOGIQUES:  À quoi est ce que l'on s'interresse lorsque l'on parle de changement d'aire de répartition chez les espèces arborescentes? Définir le concept de migration comme un long processus qui repose sur la la démographie (la survie et l'atteinte de la maturité), la dispersion (production et dissémination des graine), l'établissement (le succès de germination, la survie et la croissance des semis) (Voir Travis)
% 			- S'adapter ou changer de place 
% 			- Migration chez les arbres se traduit par une suite d'évènement de colonisation et d'extinction pour 
% 			Pour comprendre les mécanismes déclencheurs d'une migration potentielle. On ne parle pas ici d'adaptation mais de résistance et inertie du système aux changements.
% 			- Postulat Majeur: évolution - taille de la niche change pas dans le temps
% 			- Towards the edge of the distribution, growth and establishment is reduced, while growth efficiency-related mortality increases. (Thullier 2008).
% 			
% \textbf{3e.}  Concilier la théorie de la niche avec celle des métapops: une avenue pour régler ce problème ?
% 	Nouvelle génération de modèle plus mécanistiques: Démographie + dispersion
% - Théorie de la niche
% - Théorie métapop
% - Lien le concept la niche d'établissement, peristence ().
% - Reprendre le schema de Holt et replacer les équation de levins dedans
% 
% \textbf{4e.}  Description du système et transition vers le cadre méthodologique
% 
% Les conditions microclimatiques que l’on retrouve sous les espèces boréales sont différentes de celles présentes au sein de l’érablière. Au printemps, la température y est plus froide en raison de l’ombrage et par conséquent la neige y demeure plus longtemps, le sol y est plus humide et la litière est plus acide et plus fibreuse. Ainsi, même si les conditions climatiques à l’échelle de la région sont favorables à l’établissement de la régénération de l’érable à sucre, les conditions particulières retrouvées en forêt boréale pourraient nuire à sa régénération. Si c’est le cas, il est possible que l’érable à sucre ne parvient pas à s’installer en forêt boréale à la suite d’un réchauffement climatique et que cela affecte la migration vers le nord d’espèces tempérées.
% 
% - Les approches de modélisation
% - Étude des taux de migration
% - Objectif 
% 
% \section*{\uppercase{OBJECTIFS DE L'ÉTUDE}}
% - À partir des données disponibles

%  \bibliography{Master.bib}
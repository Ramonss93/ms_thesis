% L’introduction générale doit présenter la problématique des travaux de recherche, les principales dimensions méthodologiques, s'il y a lieu, et les résultats obtenus directement en rapport avec la contribution de celui qui produit la thèse ou le mémoire. Il faut bien montrer comment se situe la contribution originale de l'auteur (ou des auteurs) par rapport aux travaux cités dans la liste des références bibliographiques. Les sous-titres de l’introduction générale ne sont pas numérotés.

Depuis l'ère industrielle, la proportion des feuillues dans le paysage forestier du Québec
méridional est en constante augmentation \citep{Dupuis2011,Boucher2006}. Ce phénomène pourrait
s'amplifier d'ici la fin du XXI$^e$ siècle.  Le paysage forestier du Québec méridional est occupé en
grande majorité par la forêt tempérée qui couvre une superficie de 209 700 km$^2$
\citep{Boulay2015}. Ce biome peut être désigné comme la forêt habitée du Québec considérant qu'il se
retrouve dans la zone la plus densément peuplée de la province \citep{Doyon2009}. Depuis plusieurs
années, la forêt tempérée est confrontée à de nombreux enjeux écologiques tels que les
problématiques d'enfeuillement, la raréfaction de certaines essences, l'envahissement par d’autres
ou encore la simplification des structures internes des peuplements \citep{Varady-Szabo2008}.
Aujourd'hui, la forêt tempérée nordique doit faire face à une nouvelle problématique: celle des
changements climatiques. Elle pose plusieurs enjeux écologiques majeurs pour les aménagistes: (1)
des modifications dans la composition de la régénération post-perturbation; (2) une modulation de la
productivité forestière chez certaines espèces; (3) une modification du régime de perturbation
(p.ex. épidémies, verglas, chablis); puis enfin (4) des changements dans la répartition des espèces.
Ce mémoire porte sur ce quatrième volet. Il s'intéresse à la biogéographie et la dynamique de la
communauté de la forêt tempérée nordique dans ce contexte de changements climatiques.

\section*{Cadre conceptuel}
\addcontentsline{toc}{section}{\protect\numberline{}Cadre conceptuel}

Le climat exerce un contrôle dominant sur la distribution des espèces à l'échelle du paysage
\citep{Pearson2003b}. L'étude des registres polliniques démontrent que les fluctuations climatiques
de la période du Quaternaire ont engendré des contractions et expansions dans l'aire de distribution
des espèces \citep{Davis2001}. Aujourd'hui, l'effet des changements climatiques est déjà notable sur
la diversité végétale \citep{Walther2002a,Parmesan2006}. Considérant l'augmentation de température
de 4 à 7$^\circ$C \citep[Scénario RCP 8.5]{Climatique2015} attendue pour le Québec méridional,
plusieurs études prédisent que des espèces de la forêt tempérée (\textit{Fagus grandifolia},
\textit{Betula alleghaniensis}, \textit{Acer saccharum}, \textit{Quercus rubra}) vont étendre leur
distribution vers le nord d'ici la fin du siècle \citep{Iverson2002,Sciences2014}. Cependant, ces
changements dans la composition végétale pourraient s'avérer difficiles, considérant que les
microconditions environnementales que l’on retrouve sous les espèces boréales sont différentes de
celles présentes en forêt tempérée \citep{Barras1998,Caspersen2005}. Ainsi, même si les conditions
climatiques à l'échelle de la région sont favorables à l'établissement des espèces tempérées, les
microconditions particulières à la forêt boréale pourraient nuire à l'établissement de ces espèces \citep{DeFrenne2013,Lafleur2010}. Si c’est le cas, il est possible que la forêt tempérée
nordique ne parvienne pas à s'installer en forêt boréale à la suite d’un réchauffement climatique.

Les prédictions de changements de l'aire de répartition des espèces sont établies à l'aide de
modèles dits de distribution d'espèce. Cet ensemble hétérogène d'outil statistique repose
essentiellement sur l'établissement de corrélations entre l'occurence d'une espèce et des variables
pédoclimatiques \citep{Pearson2003b, Guisan2005a}. Malgrè le succès de ces modèles dont témoignent
l'abondante litérature, ils présentent certaines lacunes. Par exemple, ils postulent majoritairement
que les espèces sont indépendantes et donc que les facteurs biotiques (p. ex. capacité et taux de
dispersion, compétition inter-spécifique) n'affectent pas l'aire de répartition de l'éspèce
\citep{Guisan2005a,Pearson2003b}. Ils assument également que la distribution de l'espèce est en
équilibre avec le climat et qu'elle répond de manière instantanée à des changements
\citep{Austin2002}. Ils sont également statiques en excluant des processus écologiques tels que la
dispersion, la succession végétale ou encore les régimes de perturbations (le feu, l'herbivorie
etc.) \citep{Austin2002,Guisan2005a}. De part ces contraintes, les modèles de distribution d'espèces
ne sont pas appropriés pour prédire les changements d'aire de répartition des écosystèmes
forestiers.

Le modèle de diffusion de \citet{SKELLAM01061951} prédit que deux composantes sont essentielles pour
déterminer le taux asympotique de migration d'une espèce dans un environnement homogène: la
démographie (taux de croissance intrinsèque, $r$) et la capacité de dispersion ($D$) .
\citet{Svenning2014a} mettent également en lumière l'importance des interactions biotiques comme
troisième composante dans le processus de migration. Pour prédire l'aire de distribution de la forêt
tempérée, ces composantes sont essentielles puisque les arbres disposent d'une
faible capacité à se disperser, d'une forte longévité, d'un faible taux de croissance et enfin, sont
soumis à une forte compétition interspécifique à l'intérieur des peuplements
\citep{Renwick2014,Vanderwel2014}. À travers à la théorie des métapopulations
\citep{Levins1969,Holt2000,Holt2005}, ces composantes peuvent être intégrées à une suite
d'évènements de colonisation et d'extinction le long d'un gradient environnemental. Nous avons donc
développé un nouveau modèle (reposant sur cette théorie) permettant de simuler la dynamique spatiale
de colonisation de la forêt tempérée en forêt boréale afin de mieux comprendre l'effet de la
dispersion, des interactions biotiques et de la démographie sur la capacité d'expansion du biome
tempéré.


\section*{Objectifs de l'étude}
\addcontentsline{toc}{section}{\protect\numberline{}Objectifs de l'étude}

Le premier objectif de cette étude consiste à évaluer la capacité d'expansion de la forêt tempérée nordique du Québec vers le nord, d'ici la fin de ce siècle. Le deuxième objectif vise à
étudier l'effet de la dispersion, des interactions biotiques et de la démographie sur la vitesse
d'expansion (ou de contraction) de l'aire de distribution de la forêt tempérée du Québec.

Pour atteindre ces objectifs, nous utilisons un modèle d'états et de transitions (\textit{States and
Transitions model -- STM}) dérivé de la théorie des métapopulations. Il permet de simuler la
dynamique de l'écotone entre la forêt boréale et la forêt tempérée. L'écotone est classifié en
quatre communautés forestières distinctes, soit (B)oréale, (T)empérée, (M)élangée et en
(R)égénération après une perturbation -- que l'on appelle des "états". Les probabilités de
transition entre les états sont calibrées sur les transitions observées à partir des inventaires des
parcelles forestières permanentes situées aux États-Unis et au Canada (1960-2012). En résolvant le
système d'équation du modèle, nous pouvons prédire l'état attendu à l'équilibre avec les conditions
climatiques. Nous avons d'abord implémenté le modèle calibré à l'intérieur d'un automate cellulaire
afin de simuler la dynamique spatiale et temporelle de l'écotone boréal-tempéré. En utilisant le
pire scénario de changement climatique (forcage radiatif, RCP 8.5), les simulations montrent que la
forêt tempérée se déplacera seulement de 14 $\pm$ 2,0 km vers le nord alors qu'un modèle d'enveloppe
climatique prédit un déplacement de 238,79 $\pm$ 34,24 km. Les résultats des simulations mettent
également en évidence une perte de 60\% de la superficie des peuplements mélangés en faveur des
peuplements décidues d’ici la fin du siècle. Par conséquent, nous observons une diminution de la
superficie de l'écotone en 2095. Afin de comprendre l'effet des mécanismes écologiques (c.a.d
démographie, dispersion et interactions biotiques) limitant l'expansion de la forêt tempérée vers le
nord, nous avons comparé les résultats de plusieurs scénarios de simulations: (i) scénario avec
dispersion illimitée, (ii) scénario sans contrainte de démographie, (iii) résolution du modèle à
l'équilibre. La comparaison des résultats de ces simulations révèle que les interactions biotiques
et la démographie sont les facteurs les plus importants limitant la capacité d'expansion du biome de
la forêt tempérée vers le nord.

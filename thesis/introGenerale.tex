% L’introduction générale doit présenter la problématique des travaux de recherche, les principales dimensions méthodologiques, s'il y a lieu, et les résultats obtenus directement en rapport avec la contribution de celui qui produit la thèse ou le mémoire. Il faut bien montrer comment se situe la contribution originale de l'auteur (ou des auteurs) par rapport aux travaux cités dans la liste des références bibliographiques. Les sous-titres de l’introduction générale ne sont pas numérotés.

\section*{\uppercase{Mise en contexte}}
\addcontentsline{toc}{section}{\protect\numberline{}MISE EN CONTEXTE} 

Depuis l'ère industrielle, la forêt du Québec méridionale est en constante évolution (Arsenault); le paysage forestier tel que nous le connaissons aujourd'hui pourrait connaître de profondes modifications d'ici la fin du XXI$^e$ siècle. Ce paysage est occupé en grande majorité par la forêt tempérée qui couvre une superficie de 209 700 km$^2$ (MFFP, 2015). Cette forêt peut ainsi être désignée comme la forêt habitée du Québec considérant qu'elle se retrouve dans la zone la plus densément peuplée du Québec (Doyon). On y retrouve une multitude et une diversité d'activités socio-économiques tels que le tourisme, la chasse et l'acériculture et le prélèvement sylvicole. Au Québec, l'industrie forestière et l'acériculture génèrent XX et XX de dollars respectivement pour un total de XX millions d'emplois en 2015 (MFFP, Stats). La prospérité de ces activités repose sur l'intégrité écologique de ce biome forestier régionale. Sa gestion est donc primordiale, mais constitue un véritable défi de par la diversité des acteurs socio-économique, certains enjeux écologiques et les attentes de la société. Ce sont ces mêmes attentes qui ont contribué à l'adoption en XXXX d'un plan d'aménagement dit écosystémique visant à maintenir la diversité biologique et la viabilité de cet écosystème (MFFP, ).

Depuis maintenant plusieurs années, la forêt tempérée est confrontée à de nombreux enjeux écologiques tels que les problématiques d'enfeuillement, la raréfaction de certaines essences ou envahissement par d’autres, la simplification des structures internes des peuplements (Varady-Szabo). Aujourd'hui, la forêt tempérée nordique doit faire face à une nouvelle problématique qui est celle des changements climatiques. Plusieurs enjeux écologiques majeurs pour les aménagistes en découlent: (1) des modifications dans la composition de la régénération post-perturbation (2) une modulation de la productivité forestière chez certaines espèces, (3) une modification du régime de perturbation (p.ex. épidémies, verglas, chablis), puis enfin (4) des changements dans la répartition des espèces. Ce mémoire porte une attention particulière à ce quatrième aspect soit la biogéographie et la dynamique de la communauté de la forêt tempérée nordique dans ce contexte de changements climatiques.

\section*{\uppercase{BIOGÉOGRAPHIE}}
\addcontentsline{toc}{section}{\protect\numberline{}BIOGÉOGRAPHIE} 

- Théorie de la niche
- Théorie métapop
- Lien le concept la niche d'établissement, peristence ().

Les changements d'aire de répartition sont aujourd'hui documenté par des approches corrélatives. Ces outils sont limités pour prédire des changements de répartition pour des espèces possédant une forte longévité, une capacité de dispersion limité et oû la compétition est

Face à des conditions climatiques, 
- S'adapter ou changer de place 
- Migration chez les arbres se traduit par une suite d'évènement de colonisation et d'extinction pour 
Pour comprendre les mécanismes déclencheurs d'une migration potentielle. On ne parle pas ici d'adaptation mais de résistance et inertie du système aux changements.



- Les approches de modélisation
- Étude des taux de migration
- Objectif 

Ojectif:
- À partir des données disponibles
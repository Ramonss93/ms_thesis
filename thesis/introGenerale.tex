% L’introduction générale doit présenter la problématique des travaux de recherche, les principales dimensions méthodologiques, s'il y a lieu, et les résultats obtenus directement en rapport avec la contribution de celui qui produit la thèse ou le mémoire. Il faut bien montrer comment se situe la contribution originale de l'auteur (ou des auteurs) par rapport aux travaux cités dans la liste des références bibliographiques. Les sous-titres de l’introduction générale ne sont pas numérotés.

Depuis l'ère industrielle, la forêt du Québec méridionale est en constante évolution
\citep{Dupuis2011,Boucher2006}. Le paysage forestier tel que nous le connaissons aujourd'hui
pourrait connaître de profondes modifications d'ici la fin du XXI$^e$ siècle. Ce paysage est occupé
en grande majorité par la forêt tempérée qui couvre une superficie de 209 700 km$^2$
\citep{Boulay2015}. Ce biome peut ainsi être désignée comme la forêt habitée du
Québec considérant qu'elle se retrouve dans la zone la plus densément peuplée du Québec
\citep{Doyon2009}. Depuis maintenant plusieurs années, la forêt tempérée est confrontée à de
nombreux enjeux écologiques tels que les problématiques d'enfeuillement, la raréfaction de certaines
essences ou envahissement par d’autres, la simplification des structures internes des peuplements
\citep{Varady-Szabo2008}. Aujourd'hui, la forêt tempérée nordique doit faire face à une nouvelle
problématique qui est celle des changements climatiques. Plusieurs enjeux écologiques majeurs
découlent de cet problématique pour les aménagistes: (1) des modifications dans la composition de la
régénération post-perturbation; (2) une modulation de la productivité forestière chez certaines
espèces; (3) une modification du régime de perturbation (p.ex. épidémies, verglas, chablis); puis
enfin (4) des changements dans la répartition des espèces. Ce mémoire porte sur ce quatrième volet
en s'interressant à la biogéographie et la dynamique de la communauté de la forêt tempérée nordique
dans ce contexte de changements climatiques.

\section*{Cadre conceptuel}
\addcontentsline{toc}{section}{\protect\numberline{}Cadre conceptuel}

Le climat exerce un contrôle dominant sur la distribution des espèces à l'échelle du paysage
\citep{Pearson2003b}. L'étude des registres polliniques démontrent que les fluctuations climatiques
de la période du Quaternaire ont engendré des contractions et expansions dans l'aire de distribution
des espèces \citep{Davis2001}. Aujourd'hui, l'effet des changements climatiques est déjà notable sur
la diversité végétale \citep{Walther2002a,Parmesan2006}. Considérant l'augmentation de température
de 4 à 7$^\circ$C \citep[Scénario RCP 8.5]{Climatique2015} attendu pour le Québec méridionale,
plusieurs études prédisent que des espèces de la forêt tempérée (\textit{Fagus grandifolia},
\textit{Betula alleghaniensis}, \textit{Acer saccharum}, \textit{Quercus rubra}) vont étendre leur
distribution vers le nord d'ici la fin du siècle \citep{Iverson2002,Sciences2014}. Cependant, ces
changements dans la composition végétale pourrait s'avérer difficile considérant que les
microconditions environnementales que l’on retrouve sous les espèces boréales, sont différentes de
celles présentes en forêt tempérée \citep{Barras1998,Caspersen2005}. Ainsi, même si les conditions
climatiques à l'échelle de la région sont favorables à l'établissement des espèces tempérée, les
microconditions particulières à la forêt boréale, pourraient nuire à l'établissement des espèces
tempérée \citep{DeFrenne2013,Lafleur2010}. Si c’est le cas, il est possible que la forêt tempérée
nordique ne parvienne pas à s'installer en forêt boréale à la suite d’un réchauffement climatique.

L'une des approches classiques en biogéographie pour prédire les changements d'aire de répartition
d'une espèce réside dans l'utilisation de modèles d'enveloppe bioclimatique \citep{Pearson2003b}.
Ces modèles corrélatifs mettent en relation la présence et l'abscence d'une espèce avec le climat
\citep{Guisan2005a}. Ils présentent plusieurs limites puisqu'ils se basent sur la niche fondamentale
de l'espèce et ne tiennent pas compte des facteurs biotiques (p. ex. capacité et taux de dispersion,
compétition inter-spécifique) influençant son aire de répartition \citep{Guisan2005a,Pearson2003b}.
Ils assument également que la distribution de l'espèce est en équilibre avec le climat et qu'elle
répond de manière instantanée à des changements \citep{Austin2002}. Ils sont également statiques en
excluant des processus écologiques tels que la dispersion, la succession végétale ou encore les
régimes de perturbations tels que le feu ou l'herbivorie \citep{Austin2002,Guisan2005a}. En effet,
l'expansion ou la contraction des aires de répartition est un processus écologique lent caractérisé
par la faible capacité des arbres à se disperser, leurs fortes longévités, leurs faibles taux de
croissance et enfin la forte compétition interspécifique présent au sein des peuplements
\citep{Renwick2014,Vanderwel2014}. De part ces contraintes, les modèles d'enveloppe bioclimatiques
ne sont pas approprié pour prédire les changements d'aire de répartition des écosystèmes forestiers. 
- Conséquences: source-puit, persistence
- Sous-estimation de la niche fondamentale de l'espèce.
- Situtation de déséquilibre  

Les modèles de dynamique de la végétation
(\textit{Dynamic vegetation models}, DVMs) permettent de combler certaines de ces limites par des
approches plus mécanistiques \citep{Snell2014a}. Ils intègrent par exemple la phénologie des espèces
\citep{Letters2001,Morin2008}, ou encore leurs capacités de dispersion \citep{Nobis2014,Iverson2004}
et enfin peuvent ajouter une composante démographique \citep{Lischke2006a,Vanderwel2014}. Ces
modèles améliore la qualité des prédictions mais ne s'interresse pas à comparer la
contribution de chaque mécanisme écologiques (dispersion, interaction biotiques et démographie) afin
de mieux comprendre leurs effets sur la capacité de migration des espèces. 

\section*{Objectifs de l'étude}
\addcontentsline{toc}{section}{\protect\numberline{}Objectifs de l'étude}

Le premier objectif de cet étude consiste à évaluer si la forêt tempérée nordique du Québec sera en
mesure de suivre sa niche climatique d'ici la fin de ce siècle. Le deuxième objectif vise à étudier
l'effet de la dispersion, des interactions biotiques et de la démographie sur la vitesse d'expansion
(ou de contraction) de l'aire de distribution de la forêt tempérée du Québec. 

Pour atteindre ces
objectifs, nous utilisons un modèle d'états et de transitions (\textit{States and Transitions model --
STM}) dérivé de la théorie des métapopulations permettant de simuler la dynamique de l'écotone
forestier boréale et tempérée. L'écotone forestier est classifié en quatre communautés forestières
distinctes soit (B)oréales, (T)empérées, (M)élangées et en (R)égénération après une perturbation --
que l'on appelle "état". Les probabilités de transition entre les états se base sur les transitions
observés à partir des inventaires des parcelles forestières permanentes présents aux États-Unis et
au Canada (1960-2012). En utilisant le scénario RCP 8.5, les projections de la dynamique de
l'écotone forestier montrent que la forêt tempérée se déplacera seulement de 14 $\pm$ 2,0 km alors
qu'un modèle de distribution d'espèces standard prédit un déplacement de 238,79 $\pm$ 34,24 km. Ces
projection mettent également en évidence un accroissement des problématiques d’enfeuillement dans
les zones de transition d’ici la fin du siècle. Nous nous attendons à ce que la transition entre le
biome tempéré et le biome boréal soit de plus en plus abrupte, se traduisant par la diminution au
cours du temps des peuplements mixtes. L'utilisation du modèle avec un scénario de dispersion
infinie révèle que les interactions biotiques et la démographie sont les facteurs les plus
importants qui limitent la capacité d'expansion du biome de la forêt tempérée.


% Le modèle de diffusion tel permet de voir l'expansion de la forêt tempérée nordique comme un processus structuré autours de la démographie (la survie et la croissance des arbres) et la dispersion (dissemination des graines) \citep{Schurr2012,Travis2013}.  
% . À une échelle biogéographique, ces changements sont stru 
% - Processus lent structuré par le climat. 
% - Lien vers le climat
% - Lien vers la niche (persistence et établissement)
% - Lien vers les metapops (persistence et établissement résulte d'une suite d'évènement de colonization et d'extinction)
% - Transfert vers le STM
% Ces évènement peuvent être reporter à une suite d'évènements de colonisation et d'extinction à une échelle locale. Si le taux de colonization excède le taux d'extinction alors l'éspèce sera en mesure de s'établir et persister au sein du peuplement.  

% \section*{\uppercase{OBJECTIFS DE L'ÉTUDE}}
% - À partir des données disponibles
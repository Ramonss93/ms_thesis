% L’introduction générale doit présenter la problématique des travaux de recherche, les principales dimensions méthodologiques, s'il y a lieu, et les résultats obtenus directement en rapport avec la contribution de celui qui produit la thèse ou le mémoire. Il faut bien montrer comment se situe la contribution originale de l'auteur (ou des auteurs) par rapport aux travaux cités dans la liste des références bibliographiques. Les sous-titres de l’introduction générale ne sont pas numérotés.


Depuis l'ère industrielle, la forêt du québec méridionale est en constante évolution (Arsenault); le paysage forestier aue nous connaissons aujourd'hui pourrait connaitre de profondes modifications d'ici la fin du XXI$^e$ siècle. Ce paysage est occupé en grande majorité par la forêt tempérée qui couvre une superficie de 209 700 km$^2$ (MFFP, 2015). Depuis maintenant plusieurs années, la forêt tempérée est confronté à de nombreux enjeux écologiques tels que les problématiques d'enfeuillement, la raréfaction de certaines essences ou envahissement par d’autres, la simplification des structures internes des peuplements, modifiant ainsi profondément sa structure et sa composition en essence (Varady-Szabo). Aujourd'hui, la forêt tempérée du Québec méridionale doit faire face à un nouvel enjeux qui est celui des Changements Climatiques. 

La forêt tempérée nordique présente une valeur sociétaire et économique importante considérant qu'elle se retrouve là ou la majorité de la population réside (Doyon).  Elle peut ainsi être désigné comme la forêt habitée du Québec considérant la multiciplicité et la diversité des activités socio-économiques qui y sont pratiqués (randonnées, chasse, prélèvement sylvicole). L'industrie forestière constitue une des parts les plus importantes de ces activités en générant XX de dollars et XX millions d'emplois en 2015. D'un autre coté, maintenir l'intégrité écologique de cet écosystéme est primordiale afin d'assurer la continuité des services écosystémiques qui contribue à la prosperité de ces activités.
La gestion de cet écosystème revêt donc d'un véritables défi de part la multiciplicité des acteurs économique, les attentes de la société (aménagement écosystémique) et le changement dans les conditions environnementales à venir. 

Couplage changement climatiques, activités humaines (Voir GCB)
Gestion d'une cible en mouvement, complique l'aménagement écosystémique.

de part la présence d'essence commerciale de grande valeur, valeur environnemental: service écosystémique, valeur sociétaire: c'est la forêt habitée du Québec. Les changements climatiques vont boulversée le fonctionnement de cet écoystème et faconnée le paysage forestier du québec méridionale à la fin du siècle.

Pédiction: Pour préparer les acteurs socio-économique: Changements dans la régénération, modulation de la productivité, Changements dans la distribution géographique des espèces.
Compréhension: Pour comprendre les mécanismes déclencheurs d'une migration potentielle. On ne parle pas ici d'adaptation mais de résistance et inertie du système aux changements.

Les changements d'aire de répartition sont aujourd'hui documenté par des approches corrélatives. Ces outils sont limités pour prédire des changements de répartition pour des espèces possédant une forte longévité, une capacité de dispersion limité et oû la compétition est

- Les approches de modélisation
- Étude des taux de migration
- Objectif 

Ojectif:
- À partir des données disponibles
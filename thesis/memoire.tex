% ----------------------------------------------------------------------%
% Coquille pour thèses et mémoires.                                    %
% UQAR, 29 mai 2009.                                                   %
% Modifié par F. Cyr - aout 2013                                       %
%		 AC. Tassel - janvier 2014                                     %
%		 C. Rigaud - 2014-2015                                         %
% ----------------------------------------------------------------------%

% ----------------------------------------------------------------------%
% 1- Préambule.                                                        %
% ----------------------------------------------------------------------%

% Pour le dépôt initial (recto),       utiliser  oneside.
% Pour le dépôt final   (recto-verso), remplacer oneside par twoside.

\documentclass[12pt,oneside,letterpaper]{stylethese}
% \documentclass[12pt,twoside,letterpaper]{stylethese}

\raggedbottom % Évite les espaces trop gands entre chaque section.

% \usepackage{multirow}
\usepackage{natbib}
\usepackage{amsmath}
\usepackage{enumerate} % For fancy enumerate item labels.
% \usepackage{utopia}
\usepackage{txfonts,empheq}
% \usepackage{chancery}
%\usepackage{ccaption}		 %pour utiliser \legend (pour l'explication de la figure,texte plus long)
%\usepackage{pgfplots}      	 % dessiner des graphes direct dans LaTeX
%\pgfplotsset{compat=1.3}

% Sous Unix/Linux, utiliser  latin1 comme encodage.
% Sous Windows,    remplacer latin1 par ansinew. (à vérifier, moi ça marche comme ça)
% Sous MacOS,      remplacer latin1 par applemac.

% \usepackage[latin1]{inputenc}
\usepackage[utf8x]{inputenc}
\usepackage{ucs}


\usepackage{chapterbib} % <----- for bibliography per chapter
%\usepackage[duplicate]{chapterbib}
%\usepackage{url}

%% graphics packages
\usepackage{graphicx}
\usepackage{tikz}
\usepackage{pgfplots}
\usepackage{pgfplotstable}
\usepackage{xcolor}
\usepackage{pdflscape}
\usepackage{wrapfig}
\usepackage{amsmath,amsfonts,amssymb}
\usepackage{diagbox}
\usepackage{titling}


% Ajout Cyril :
\usepackage[linktocpage=true,linkcolor=blue,citecolor=blue,colorlinks=true,urlcolor=blue]{hyperref}
\usepackage[english,french]{cleveref}
\usepackage{upgreek}
\usepackage{textcomp}
\usepackage{tabularx}
\usepackage{longtable}
\usepackage{ltxtable}
\usepackage{pdflscape}
\usepackage{booktabs}
\usepackage{afterpage}
\usepackage{floatpag}
\usepackage[babel=true]{csquotes}
\usepackage{pifont}
\usepackage{soulutf8}
\usepackage{caption}
\captionsetup{singlelinecheck=false}

\setlength{\LTcapwidth}{1.55\linewidth}

%Marge inférieure avec note de bas de page
\setlength{\footnotesep}{1cm}

%Régler le problème des Et/And entre deux auteurs (voir le fichier "tutoriel bst")
\newcommand*{\andname}{and}
      \addto \captionsenglish {\renewcommand*{\andname}{and}}
      \addto \captionsfrench  {\renewcommand*{\andname}{et}}

%Intitulé des figures en français
\addto\captionsfrench{\renewcommand{\figurename}{Figure}}

%Intitulé des tableaux en français
\addto\captionsfrench{\renewcommand{\tablename}{Tableau}}

% Interligne 1 et 1/2.
\setstretch{1.5}

%%%%% Steve section
\definecolor{States}{RGB}{51,51,51}
\usetikzlibrary{shapes, arrows, positioning}
\usepackage{pdflscape}


% Pour créer un index (optionnel).

\makeindex

% Info sur la thèse (Titre, auteur, etc.)
\Titre{BIOGÉOGRAPHIE ET DYNAMIQUE DE LA FORÊT TEMPÉRÉE NORDIQUE DANS UN CONTEXTE DE CHANGEMENTS CLIMATIQUES}
\Auteur{Steve Vissault}
\Faculte{programme de maîtrise en gestion de la faune et de ses habitats}
\Diplome{maître ès sciences}
\Date{Mai}{2016}

% Utiliser \These pour une thèse, ou \Memoire pour un mémoire.
\Memoire

% Utiliser      \Chapitres pour une thèse (ou mémoire) traditionnelle.
% Remplacer par \Articles  pour une thèse (ou mémoire) par articles.
\Articles
% \Chapitres

\begin{document}
\pdfstringdefDisableCommands{%
\let\MakeUppercase\relax}
% ----------------------------------------------------------------------%
% 2- Liminaires de la thèse.                                           %
% ----------------------------------------------------------------------%

%----------------------------------------------------------------------%
% Liminaires de la thèse.                                              %
% UQAR septembre 2013                                                  %
% ---------------------------------------------------------------------%

% ----------------------------------------------------------------------%
% 1- Page titre.                                                        %
% ----------------------------------------------------------------------%

\Pagetitre
\cleardoublepage
% ----------------------------------------------------------------------%
% inclusions qui pourraient mériter d'être incluses dans le .cls
% (commentez si non-nécessaire)
% 1.1 - Composition du Jury.                                           %
\thispagestyle{empty}

\null
\vfill
\noindent \textbf{Composition du jury:}\\
\vspace{1cm}

\begin{singlespace}
  \noindent \textbf{Dominique Arsenault, président du jury, Université du Québec à Rimouski}\\

  \noindent \textbf{Dominique Gravel, directeur de recherche, Université du Québec à Rimouski}\\

  \noindent \textbf{Matthew Talluto, codirecteur de recherche, Université Joseph Fourier}\\

  \noindent \textbf{Isabelle Boulangeat, codirecteur de recherche, Université du Québec à Rimouski}\\

  \noindent \textbf{[Prénom Nom], examinateur externe, [Université d’attache]}\\
\end{singlespace}

\vspace{2cm}
\noindent Dépôt initial le [date mois année]
\hspace{3cm}
Dépôt final le [date mois année]


\cleardoublepage

% % 1.2 - Avertissement biblio.
\input{avertissement.tex}
% % 1.3 - Dedicace.
\thispagestyle{empty}

\begin{minipage}[l]{0.45\textwidth}

\end{minipage}%
\hfill
\begin{minipage}[r]{0.5\textwidth}
\begin{quotation}
\begin{doublespace}

\textit{À mes parents et tous ceux qui ont été patient...}


\end{doublespace}
\end{quotation}
\end{minipage}%

\cleardoublepage

% ----------------------------------------------------------------------%


% ----------------------------------------------------------------------%
% 2- Remerciements.                                                    %
% ----------------------------------------------------------------------%

\remerciements

\selectlanguage{frenchb}

Je tiens dans un premier temps à remercier mon directeur Dominique Gravel pour m'avoir donné l'opportunité de réaliser cette maitrise. Je tiens également à remercier Isabelle Boulangeat et Matthew Talluto pour leurs implications et .

% ----------------------------------------------------------------------%
% 3- Avant-propos.                                                     %
% ----------------------------------------------------------------------%

\avantpropos

\selectlanguage{frenchb}

[Cette page est facultative; l’éliminer si elle n’est pas utilisée. L’avant-propos ne doit pas être confondu avec l'introduction. Il n’est pas d’ordre scientifique alors que l’introduction l’est. Il s’agit d'un discours préliminaire qui permet notamment à l'auteur d'exposer les raisons qui l'ont amené à étudier le sujet choisi, le but qu'il veut atteindre, ainsi que les possibilités et les limites de son travail. On peut inclure les remerciements à la fin de ce texte au lieu de les présenter sur une page distincte.]

% ----------------------------------------------------------------------%
% 4- Resume/Abstract                                                           %
% ----------------------------------------------------------------------%

\resume
\begin{singlespace}

  De nombreuses espèces ne migrent pas assez vite pour suivre la rapidité des changements climatiques. Les arbres sont bien connus pour éprouver de longs délais dans leurs réponses au climat parce qu'ils sont sessiles, longévive et dispose de faible capacité de dispersion. Les approches actuelles pour prédire l'aire de répartition future des espèces ne peuvent pas tenir compte de ces particularités propre aux écosystèmes forestiers car ils assument une dispersion infinie et une réponse instantanée aux changements climatiques. À travers cette étude, nous proposons une nouvelle approche de modélisation basée sur la théorie des métapopulations pour tenir compte de la dispersion limitée, des interactions biotiques et de la démographie de la forêt tempérée. Notre objectif est d'évaluer si la forêt tempérée d'Amérique du Nord-Est sera en mesure de suivre son optimum climatique d'ici la fin du siècle. Les transitions entre les états sont calibrés à partir de plusieurs enquêtes sur les parcelles forestières à long terme des États United- et au Canada. Nous constatons que, même si les modèles de distribution des espèces classiques prédisaient un déplacement vers le nord de la distribution de la forêt tempérée de 328 $ \pm $ 28,4 kms, la forêt tempérée sera à peine bouger 14 $\pm $ 2,0 km dans la forêt boréale à la fin de ce siècle. Nous constatons également que la plupart des transitions attendues sera la conversion du mélange à des peuplements purs tempérées. Une comparaison avec un scénario de dispersion infinie révèle que les interactions biotiques et la dynamique de remplacement des peuplements sont les facteurs les plus signifcatifs limitant le taux des arbres forestiers de migration. Nous concluons que la forêt tempérée a une faible résilience au changement climatique en raison de leur faible démographie et les interactions compétitives avec des arbres résidents.

  \begin{quote}
    Mots clés: [Inscrire ici 5 à 10 mots clés]
  \end{quote}
\end{singlespace}
\cleardoublepage

\abstract
\begin{singlespace}

  Many species are not migrating fast enough to keep pace with the rapidly changing climate. Trees are well known to experience long time lags in their migration responses because they are sessile, long-lived and have a relatively short dispersal ability. Actual approaches to forecast range shifts under climate change, such as Species Distribution Models, cannot account for the particularities of forest ecosystems because they assume infinite dispersal and instantaneous response to climate change. Here, we propose a new modelling approach based on metapopulation theory to account for dispersal limitations, biotic interactions and the demography of the temperate forest. Our objective is to assess if the North-Eastern American temperate forest will be able to track its climatic optimum by the end of this century. Transitions among states are calibrated from several long-term forest plots surveys from United States and Canada. We find that even if standard species distribution models would predict a northward shift of the temperate forest distribution of 328$\pm$28.4 km, the temperate forest will barely move 14$\pm$2.0 km into the boreal forest at the end of this century. We also find than most of the expected transitions will be the conversion from mixed to pure temperate stands. A comparison with an infinite dispersal scenario reveals that biotic interactions and stand replacement dynamics are the most significant factors limiting migration rate of forest trees. We conclude that the temperate forest has a low resilience to climate change because of their low demography and competitive interactions with resident trees.
  
  \begin{quote}
    Keywords: [Inscrire ici 5 à 10 mots clés]
  \end{quote}
\end{singlespace}
\cleardoublepage

% ----------------------------------------------------------------------%
% 5- Table des matières.                                               %
% ----------------------------------------------------------------------%

\tabledesmatieres

% ----------------------------------------------------------------------%
% 6- Liste des tableaux.                                               %
% ----------------------------------------------------------------------%

\listedestableaux

% ----------------------------------------------------------------------%
% 7- Table des matières.                                               %
% ----------------------------------------------------------------------%

\listedesfigures


% ----------------------------------------------------------------------%
% Fin des liminaires.                                                  %
% ----------------------------------------------------------------------%

\cleardoublepage


% ----------------------------------------------------------------------%
% 3- Corps de la thèse.                                                %
% ----------------------------------------------------------------------%

\debutcorps
\cleardoublepage
% Utiliser      chapitres.tex pour une thèse (ou mémoire) traditionnelle.
% Remplacer par articles.tex  pour une thèse (ou mémoire) par articles.


\introduction
\selectlanguage{french}
% L’introduction générale doit présenter la problématique des travaux de recherche, les principales dimensions méthodologiques, s'il y a lieu, et les résultats obtenus directement en rapport avec la contribution de celui qui produit la thèse ou le mémoire. Il faut bien montrer comment se situe la contribution originale de l'auteur (ou des auteurs) par rapport aux travaux cités dans la liste des références bibliographiques. Les sous-titres de l’introduction générale ne sont pas numérotés.

\section*{Mise en contexte}
\addcontentsline{toc}{section}{\protect\numberline{}Mise en Contexte}

Depuis l'ère industrielle, la forêt du Québec méridionale est en constante évolution (Arsenault); le
paysage forestier tel que nous le connaissons aujourd'hui pourrait connaître de profondes
modifications d'ici la fin du XXI$^e$ siècle. Ce paysage est occupé en grande majorité par la forêt
tempérée qui couvre une superficie de 209 700 km$^2$ (MFFP, 2015). Cette forêt peut ainsi être
désignée comme la forêt habitée du Québec considérant qu'elle se retrouve dans la zone la plus
densément peuplée du Québec (Doyon). On y retrouve une multitude et une diversité d'activités
socio-économiques tels que le tourisme, la chasse et l'acériculture et le prélèvement sylvicole. Au
Québec, l'industrie forestière et l'acériculture génèrent XX et XX de dollars respectivement pour un
total de XX millions d'emplois en 2015 (MFFP, Stats). La prospérité de ces activités repose sur
l'intégrité écologique de ce biome forestier régionale. Sa gestion est donc primordiale, mais
constitue un véritable défi de par la diversité des acteurs socio-économique, certains enjeux
écologiques et les attentes de la société. Ce sont ces mêmes attentes qui ont contribué à l'adoption
en XXXX d'un plan d'aménagement écosystémique visant à maintenir la diversité biologique et la
viabilité de cet écosystème (MFFP, ).

Depuis maintenant plusieurs années, la forêt tempérée est confrontée à de nombreux enjeux
écologiques tels que les problématiques d'enfeuillement, la raréfaction de certaines essences ou
envahissement par d’autres, la simplification des structures internes des peuplements
(Varady-Szabo). Aujourd'hui, la forêt tempérée nordique doit faire face à une nouvelle problématique
qui est celle des changements climatiques. Plusieurs enjeux écologiques majeurs découlent de cet
problématique pour les aménagistes: (1) des modifications dans la composition de la régénération
post-perturbation; (2) une modulation de la productivité forestière chez certaines espèces; (3) une
modification du régime de perturbation (p.ex. épidémies, verglas, chablis); puis enfin (4) des
changements dans la répartition des espèces. Ce mémoire porte sur ce quatrième volet en s'interressant à la biogéographie et la dynamique de la communauté de la forêt tempérée
nordique dans ce contexte de changements climatiques (c.a.d. son écotone).

\section*{Cadre conceptuel}
\addcontentsline{toc}{section}{\protect\numberline{}Cadre conceptuel}

Les forêts ont déjà connu plusieurs épisodes de changements climatiques. L'étude des registres
polliniques démontrent que ces fluctuations climatiques passées ont engendré des contractions et
expansions dans l'aire de distribution des espèces (e.g. Davis and Shaw 2001). Aujourd'hui, l'effet
des changements climatiques est déjà observable sur la diversité végétale
\citep{Walther2002a,Parmesan2006}. Considérant l'augmentation de température de 4 à 7$^\circ$C
\citep[Scénario RCP 8.5]{Climatique2015} attendu pour le Québec méridionale, plusieurs études
prédisent que des espèces de la forêt tempérée (\textit{Fagus grandifolia}, \textit{Betula
alleghaniensis}, \textit{Acer saccharum}, \textit{Quercus rubra}) vont étendre leur distribution
vers le nord d'ici la fin du siècle \citep{Iverson2002,Sciences2014}. Cependant, ce changement dans
la composition végétale pourrait s'avérer difficile considérant que les conditions microclimatiques
que l’on retrouve sous les espèces boréales sont différentes de celles présentes en forêt tempérée.
Au printemps, la température y est plus froide en raison de l'ombrage et par conséquent la neige y
demeure plus longtemps, le sol y est plus humide et la litière est plus acide et plus fibreuse.
Ainsi, même si les conditions climatiques à l'échelle de la région sont favorables à l'établissement
des espèces tempérée, les microconditions particulières retrouvées en forêt boréale pourraient nuire à
sa régénération \citep{DeFrenne2013,Lafleur2010}. Si c’est le cas, il est possible que la forêt
tempérée nordique ne parvienne pas à s'installer en forêt boréale à la suite d’un réchauffement
climatique et que cela contraigne l'expansion vers le nord de la distribution des espèces tempérées.

Pour étudier les changemens d'aire de répartition d'une espèce, les modèles communément utilisé en
biogéographie sont les mdèles d'enveloppes bioclimatiques. Ces derniers reposent sur des approches
corrélatives mettant en relation la présence et l'abscence d'une espèce avec le climat. Ces outils
présentent plusieurs limites puisqu'ils se basent sur la niche fondamentale et ne tiennent donc pas
compte des facteurs biotiques (p. ex. capacité et taux de dispersion, compétition inter-spécifique)
influençant l'aire de répartition des espèces [Guisan2005, Pearson2003]. Ces facteurs jouent
pourtant un rôle prépondérant dans la dynamique d’un écosystème [Guisan2005, Araujo2007,
Pearson2003]. Ces modèles assument également que la végétation ne présente pas de déséquilibre avec
le climat et qu'elle répond de manière instantanée aux fluctuations climatiques [Austin2002]. Ils
sont donc statiques et ne contiennent aucune composante dynamique (p. ex. les processus de
dispersion, succession ou encore de perturbations tels que le feu ou le broutage) [Guisan2005,
Austin2002].

% Cette relation permet de dégager la niche fondamentale de l'espèce  communément appelé modèle d'enveloppe bioclimatiques. Parmis les approches mécanistiques, des modèles tels que Phenofit \citep{Letters2001} permettent de prendre en considération la phénologie des espèces en . 

%
% - Limite des arbres à la migration
%
% - Outils disponible
%
% Pour comprendre et prédire, ces changements
% . Dessiné par des contraintes physiologiques à échelle régionnale mais à plus fine échelle des contraintes biotiques et abiotiques
% - Est-ce que la forêt tempérée suivra le rythme ?
% Depuis maintenant 25 ans, l'écotone entre la forêt boréal et la forêt tempérée nordique fait l'objet d'une attention particulière \citep{Goldblum2010}. Cette
% Rapidité des changements. Connaitre la vélocité de ses changements est primodiale pour améliorer notre capacité à adapter nos pratiques sylvicole.
%
% Migration pour des espèces végétale, en quoi ca consiste ?
%
% Outils disponibles pour étudier la migration
%
% Pour comprendre ces changements dans la distribution,
% l'une des approches classiques centrée sur l'espèce consiste à produire des modèles dit corrélatif.
% se rapproche de la niche fondamentale
% - Thuillier + Guisan (2005)
% Puisan (2000)
%
% Pour prédire des changemens d'aire de répartition, deux grandes familles de modèles éxistent:
% 	- SDM (BIOMOD2, enveloppe bioclimatique)
% 	- DVM tente de corriger par des approches plus mécanistiques en tenant compte de la phénologie des espèces (PhenoFit, Morin), se capacité à se disperser (TreeMig, Liscke, SHIFT), démographie (CAIN, Vanderwhel).
% 	- En couplant, SDM avec modèles de migration CAIN, SHIFT.
% - Certains postulats ne sont pas approprié lorsque l'on tente de modéliser le réponse des arbres.
% - Conclusion du paragraphe, nécéssité de se dirger vers une nouvelle générationde modèle
%
% Les changements d'aire de répartition sont aujourd'hui documenté par des approches corrélatives. Ces outils sont limités pour prédire des changements de répartition pour des espèces possédant une forte longévité, une capacité de dispersion limité et oû la compétition est
%
% Les outils disponibles et limites pour étudier ce phénomène. Finir avec l'importance d'intégrer la dispersion et la démographie
%
% Les modèles d’enveloppe bioclimatique possèdent plusieurs limites. Ces outils se basent sur la niche fondamentale d’une espèce et ne tiennent donc pas compte des facteurs biotiques (p. ex. capacité et taux de dispersion, compétition inter-spécifique) influençant l’aire de répartition de cette dernière [Guisan2005, Pearson2003]. Ces facteurs jouent pourtant un rôle prépondérant dans la dynamique d’un écosystème [Guisan2005, Araujo2007, Pearson2003]. De plus, ces modèles statistiques assument que la végétation a atteint partiellement ou complètement son point d’équilibre avec le climat [Austin2002]. Ils sont donc statiques et ne contiennent aucune composante dynamique (p. ex. les processus de dispersion, succession ou encore de perturbations tels que le feu ou le broutage) [Guisan2005, Austin2002].
%
% 	Limites:
% 	- Réponse instantanée
% 	- Importance des processus démographique et de dispersion (Holt,2005)
% 			- Conduit vers
%
% \textbf{1er}. RECENTRER SUR LE PROBLÈME ÉCOLOGIQUES:  À quoi est ce que l'on s'interresse lorsque l'on parle de changement d'aire de répartition chez les espèces arborescentes? Définir le concept de migration comme un long processus qui repose sur la la démographie (la survie et l'atteinte de la maturité), la dispersion (production et dissémination des graine), l'établissement (le succès de germination, la survie et la croissance des semis) (Voir Travis)
% 			- S'adapter ou changer de place
% 			- Migration chez les arbres se traduit par une suite d'évènement de colonisation et d'extinction pour
% 			Pour comprendre les mécanismes déclencheurs d'une migration potentielle. On ne parle pas ici d'adaptation mais de résistance et inertie du système aux changements.
% 			- Postulat Majeur: évolution - taille de la niche change pas dans le temps
% 			- Towards the edge of the distribution, growth and establishment is reduced, while growth efficiency-related mortality increases. (Thullier 2008).
%
% \textbf{3e.}  Concilier la théorie de la niche avec celle des métapops: une avenue pour régler ce problème ?
% 	Nouvelle génération de modèle plus mécanistiques: Démographie + dispersion
% - Théorie de la niche
% - Théorie métapop
% - Lien le concept la niche d'établissement, peristence ().
% - Reprendre le schema de Holt et replacer les équation de levins dedans
%
% \textbf{4e.}  Description du système et transition vers le cadre méthodologique
%
% 
%
% - Les approches de modélisation
% - Étude des taux de migration
% - Objectif
%
% \section*{\uppercase{OBJECTIFS DE L'ÉTUDE}}
% - À partir des données disponibles

%  \bibliography{Master.bib}

\cleardoublepage


\selectlanguage{english}

\chapter{La démographie, une contrainte à l'expansion de la forêt tempérée vers le Nord}

\section{Résumé en français du premier article}

De nombreuses espèces ne migrent pas assez vite pour suivre la rapidité des changements climatiques.
Les arbres sont bien connus pour éprouver de longs délais dans leurs réponses au climat parce qu'ils sont sessiles, possèdent une forte longévité et disposent de faible capacité de dispersion. Les approches actuelles pour prédire l'aire de répartition future des espèces, telles que les modèles d'enveloppe climatique, ne peuvent pas tenir compte de ces particularités propres aux écosystèmes forestiers, car ils assument une dispersion infinie et une réponse instantanée aux changements climatiques. Nous proposons une nouvelle approche de modélisation basée sur la théorie des métapopulations pour tenir compte de cette capacité limitée de dispersion, des interactions biotiques et de la démographie propre à la forêt tempérée nordique du nord-est de l'Amérique du Nord. Notre objectif est d'évaluer si ce biome forestier sera en mesure de suivre sa niche climatique d'ici la fin de ce siècle. Nous avons effectué des simulations de l'écotone entre la forêt boréale et tempérée en utilisant un modèle d'états et de transitions (STM), dans lequel les communautés forestières sont classées dans 4 états: boréales, tempérées, mélangées et en régénération après une perturbation. Les transitions entre les états sont calibrées à partir des inventaires des parcelles permanentes présents aux États-Unis et au Canada. Les résultats des simulations du modèle indiquent que la forêt tempérée se déplacera seulement de 14 $\pm$ 2,0 km alors qu'un modèle de distribution d'espèces standard prédit un déplacement de 238,79 $\pm$ 34,24 km. Les simulations de l'écotone forestier mettent également en évidence que la majorité des transitions attendues seront une conversion des peuplements mixtes vers des peuplements purement décidus. L'utilisation du modèle avec un scénario de dispersion infinie révèle que les interactions biotiques et la démographie sont les facteurs les plus importants qui limitent la capacité d'expansion du biome de la forêt tempérée. En conclusion, la forêt tempérée possède une faible résilience au changement climatique en raison de sa lente démographie et des fortes interactions compétitives avec les espèces boréales.

Ce premier article, intitulé \enquote{\textit{Slow demography constrains the North-Eastern Temperate
Forest expansion under Climate Change}}, fut corédigé par moi-même ainsi que mon Professeur
Dominique Gravel et mes deux cosuperviseurs, Matthew Talutto (Ph.D) et Isabelle Boulangeat (Ph.D).
L'article présenté sera soumis à \enquote{\textit{Global Change Biologie}} pour évaluation par mes
pairs à la fin de l'été 2016. Cet article constitue l'un des volets du projet stratégique QUICC-FOR,
financé par le CRSNG, qui vise à cartographier et quantifier les conséquences des changements
climatiques sur les forêts de l'Est de l'Amérique du Nord. Ma contribution en tant qu'auteur se
résume en cinq points: (i) effectuer un état de la littérature; (ii) conceptualiser le modèle et
l’implémenter grâce au langage de programmation C; (iii) créer une base de données nécessaire à la
calibration et la validation du modèle; (iv) effectuer le post-traitement et l'analyse des
simulations; (v) rédiger l'article. Dominique Gravel est à l'origine de l'idée du projet et a aidé à
la conceptualisation, la validation du modèle et la révision du manuscrit. Matthew Talluto est
responsable de la calibration bayésienne avec la méthode MCMC (\textit{Monte Carlo Markov Chain}).
Il a également contribué à l'implémentation du modèle en C ainsi qu'à la révision du manuscrit.
Isabelle Boulangeat est responsable de l'estimation des paramètres par maximum de vraisemblance
nécessaire à l'initialisation du MCMC. Elle a également contribué à la révision du manuscrit.
L'ensemble de mon équipe d'encadrement a fourni une assistance technique inestimable dans la plupart
des étapes scientifiques nécessaires à l'obtention de ces résultats.

Les résultats présentés ici reposent sur les paramètres issus la calibration par maximum de vraisemblance. De nouvelles simulations vont être amorcées en utilisant la distribution postérieure du MCMC afin d'obtenir une meilleure estimation de l'incertitude dans les projections. Les résultats de cet article ont été présentés à deux congrès internationaux et deux congrès provinciaux sous forme d'une affiche et d'une conférence. La conférence s'intitulait \enquote{\textit{Difficult migration of temperate tree species in boreal forest under climate change?}}, présenté au 9$^e$ colloque du Centre d'étude de la Forêt (CEF) en avril 2015 et au 7$^e$ congrès eCANUSA sur les sciences forestières en octobre 2014. Enfin, l'affiche portait le titre \enquote{\textit{A state transition model to investigate what constrains the northward migration of the temperate forest}} et a fait l'objet d'une présentation au colloque du Centre des sciences de la biodiversité du Québec (CSBQ) en octobre 2015 et au 9$^e$ congrès IALE en écologie du paysage en juillet 2015. L'ensemble de ces travaux peuvent être téléchargés à partir de mon site internet personnel (\url{http://steveviss.github.io/paper/}). En parallèle à cette maîtrise, je me suis impliqué à titre de coauteur dans l'un des chapitres de la thèse de Christian Marchese (UQAR) portant sur la phénologie du phytoplancton dans la région arctique de la mer de Baffin. Cette collaboration a aboutie à un article intitulé \enquote{\textit{Changes in phytoplankton bloom phenology in the NOW polynya region: a response to changing environmental conditions}}, soumis en mars 2016 dans la revue \textit{Polar Biology}.




\newpage

\section{Slow demography constrains the North-Eastern Temperate Forest expansion under Climate Change}

\textbf{AUTHORSHIP}

\cleardoublepage

\conclusion
\selectlanguage{french}
[C’est dans cette section qu’est mise en évidence la portée de l’étude ainsi que les liens entre les articles ou autres textes et une ouverture sur les perspectives de recherche dans le domaine concerné. On y fait état des limites de la recherche et on y propose, le cas échéant, des pistes nouvelles pour de futures recherches ou des façons de développer de nouvelles applications. La conclusion ne doit pas présenter de nouveaux résultats ni de nouvelles interprétations. Elle doit être rédigée de manière à faire ressortir la cohérence de la démarche.]


1. Principaux résultats

2. Limites du modèle

- Change in species assenblage
- Evolution of the niche, taille de la niche peut changer dans le temps

3. Implication des résultats dans un contexte d'aménagement

- Adaptation aux CC
- Aménager pour une cible en mouvement
\cleardoublepage


% ----------------------------------------------------------------------%
% 5 - Bibliographie.                                                    %
% ----------------------------------------------------------------------%

\begin{singlespace}
  \makeatletter
  \phantomsection\addcontentsline{toc}{chapter}{\MakeUppercase{\@references}}
  \makeatother
  \selectlanguage{english}
  \bibliographystyle{apalike} % Ici éditer le style
  \bibliography{/home/steve/Dropbox/Bibtex/Master} % Ici mettre le nom de la biblio, ici mylib.bib
\end{singlespace}

% ----------------------------------------------------------------------%
% Fin du document.                                                     %
% ----------------------------------------------------------------------%

\end{document}

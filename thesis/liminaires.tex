%----------------------------------------------------------------------%
% Liminaires de la thèse.                                              %
% UQAR septembre 2013                                                  %
% ---------------------------------------------------------------------%

% ----------------------------------------------------------------------%
% 1- Page titre.                                                        %
% ----------------------------------------------------------------------%

\Pagetitre
\cleardoublepage
% ----------------------------------------------------------------------%
% inclusions qui pourraient mériter d'être incluses dans le .cls
% (commentez si non-nécessaire)
% 1.1 - Composition du Jury.                                           %
\thispagestyle{empty}

\null
\vfill
\noindent \textbf{Composition du jury:}\\
\vspace{1cm}

\begin{singlespace}
  \noindent \textbf{Dominique Arsenault, président du jury, Université du Québec à Rimouski}\\

  \noindent \textbf{Dominique Gravel, directeur de recherche, Université du Québec à Rimouski}\\

  \noindent \textbf{Matthew Talluto, codirecteur de recherche, Université Joseph Fourier}\\

  \noindent \textbf{Isabelle Boulangeat, codirecteur de recherche, Université du Québec à Rimouski}\\

  \noindent \textbf{[Prénom Nom], examinateur externe, [Université d’attache]}\\
\end{singlespace}

\vspace{2cm}
\noindent Dépôt initial le [date mois année]
\hspace{3cm}
Dépôt final le [date mois année]


\cleardoublepage

% % 1.2 - Avertissement biblio.
\input{avertissement.tex}
% % 1.3 - Dedicace.
\thispagestyle{empty}

\begin{minipage}[l]{0.45\textwidth}

\end{minipage}%
\hfill
\begin{minipage}[r]{0.5\textwidth}
\begin{quotation}
\begin{doublespace}

\textit{À mes parents et tous ceux qui ont été patient...}


\end{doublespace}
\end{quotation}
\end{minipage}%

\cleardoublepage

% ----------------------------------------------------------------------%


% ----------------------------------------------------------------------%
% 2- Remerciements.                                                    %
% ----------------------------------------------------------------------%

\remerciements

\selectlanguage{frenchb}

Je tiens dans un premier temps à remercier mon directeur Dominique Gravel pour m'avoir donné l'opportunité de réaliser cette maitrise. Je tiens également à remercier Isabelle Boulangeat et Matthew Talluto pour leurs implications et .

% ----------------------------------------------------------------------%
% 3- Avant-propos.                                                     %
% ----------------------------------------------------------------------%

\avantpropos

\selectlanguage{frenchb}

[Cette page est facultative; l’éliminer si elle n’est pas utilisée. L’avant-propos ne doit pas être confondu avec l'introduction. Il n’est pas d’ordre scientifique alors que l’introduction l’est. Il s’agit d'un discours préliminaire qui permet notamment à l'auteur d'exposer les raisons qui l'ont amené à étudier le sujet choisi, le but qu'il veut atteindre, ainsi que les possibilités et les limites de son travail. On peut inclure les remerciements à la fin de ce texte au lieu de les présenter sur une page distincte.]

% ----------------------------------------------------------------------%
% 4- Resume/Abstract                                                           %
% ----------------------------------------------------------------------%

\resume
\begin{singlespace}

  De nombreuses espèces ne migrent pas assez vite pour suivre la rapidité des changements climatiques. Les arbres sont bien connus pour éprouver de longs délais dans leurs réponses au climat parce qu'ils sont sessiles, longévive et dispose de faible capacité de dispersion. Les approches actuelles pour prédire l'aire de répartition future des espèces ne peuvent pas tenir compte de ces particularités propre aux écosystèmes forestiers car ils assument une dispersion infinie et une réponse instantanée aux changements climatiques. À travers cette étude, nous proposons une nouvelle approche de modélisation basée sur la théorie des métapopulations pour tenir compte de la dispersion limitée, des interactions biotiques et de la démographie de la forêt tempérée. Notre objectif est d'évaluer si la forêt tempérée d'Amérique du Nord-Est sera en mesure de suivre son optimum climatique d'ici la fin du siècle. Les transitions entre les états sont calibrés à partir de plusieurs enquêtes sur les parcelles forestières à long terme des États United- et au Canada. Nous constatons que, même si les modèles de distribution des espèces classiques prédisaient un déplacement vers le nord de la distribution de la forêt tempérée de 328 $ \pm $ 28,4 kms, la forêt tempérée sera à peine bouger 14 $\pm $ 2,0 km dans la forêt boréale à la fin de ce siècle. Nous constatons également que la plupart des transitions attendues sera la conversion du mélange à des peuplements purs tempérées. Une comparaison avec un scénario de dispersion infinie révèle que les interactions biotiques et la dynamique de remplacement des peuplements sont les facteurs les plus signifcatifs limitant le taux des arbres forestiers de migration. Nous concluons que la forêt tempérée a une faible résilience au changement climatique en raison de leur faible démographie et les interactions compétitives avec des arbres résidents.

  \begin{quote}
    Mots clés: [Inscrire ici 5 à 10 mots clés]
  \end{quote}
\end{singlespace}
\cleardoublepage

\abstract
\begin{singlespace}

  Many species are not migrating fast enough to keep pace with the rapidly changing climate. Trees are well known to experience long time lags in their migration responses because they are sessile, long-lived and have a relatively short dispersal ability. Actual approaches to forecast range shifts under climate change, such as Species Distribution Models, cannot account for the particularities of forest ecosystems because they assume infinite dispersal and instantaneous response to climate change. Here, we propose a new modelling approach based on metapopulation theory to account for dispersal limitations, biotic interactions and the demography of the temperate forest. Our objective is to assess if the North-Eastern American temperate forest will be able to track its climatic optimum by the end of this century. Transitions among states are calibrated from several long-term forest plots surveys from United States and Canada. We find that even if standard species distribution models would predict a northward shift of the temperate forest distribution of 328$\pm$28.4 km, the temperate forest will barely move 14$\pm$2.0 km into the boreal forest at the end of this century. We also find than most of the expected transitions will be the conversion from mixed to pure temperate stands. A comparison with an infinite dispersal scenario reveals that biotic interactions and stand replacement dynamics are the most significant factors limiting migration rate of forest trees. We conclude that the temperate forest has a low resilience to climate change because of their low demography and competitive interactions with resident trees.
  
  \begin{quote}
    Keywords: [Inscrire ici 5 à 10 mots clés]
  \end{quote}
\end{singlespace}
\cleardoublepage

% ----------------------------------------------------------------------%
% 5- Table des matières.                                               %
% ----------------------------------------------------------------------%

\tabledesmatieres

% ----------------------------------------------------------------------%
% 6- Liste des tableaux.                                               %
% ----------------------------------------------------------------------%

\listedestableaux

% ----------------------------------------------------------------------%
% 7- Table des matières.                                               %
% ----------------------------------------------------------------------%

\listedesfigures


% ----------------------------------------------------------------------%
% Fin des liminaires.                                                  %
% ----------------------------------------------------------------------%

\cleardoublepage

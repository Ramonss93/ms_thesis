%----------------------------------------------------------------------%
% Liminaires de la thèse.                                              %
% UQAR septembre 2013                                                  %
% ---------------------------------------------------------------------%

% ----------------------------------------------------------------------%
% 1- Page titre.                                                        %
% ----------------------------------------------------------------------%

\Pagetitre
\cleardoublepage
% ----------------------------------------------------------------------%
% inclusions qui pourraient mériter d'être incluses dans le .cls
% (commentez si non-nécessaire)
% 1.1 - Composition du Jury.                                           %
\thispagestyle{empty}

\null
\vfill
\noindent \textbf{Composition du jury:}\\
\vspace{1cm}

\begin{singlespace}
  \noindent \textbf{Dominique Arsenault, président du jury, Université du Québec à Rimouski}\\

  \noindent \textbf{Dominique Gravel, directeur de recherche, Université du Québec à Rimouski}\\

  \noindent \textbf{Matthew Talluto, codirecteur de recherche, Université Joseph Fourier}\\

  \noindent \textbf{Isabelle Boulangeat, codirecteur de recherche, Université du Québec à Rimouski}\\

  \noindent \textbf{[Prénom Nom], examinateur externe, [Université d’attache]}\\
\end{singlespace}

\vspace{2cm}
\noindent Dépôt initial le [date mois année]
\hspace{3cm}
Dépôt final le [date mois année]


\cleardoublepage

% % 1.2 - Avertissement biblio.
\input{avertissement.tex}
% % 1.3 - Dedicace.
\thispagestyle{empty}

\begin{minipage}[l]{0.45\textwidth}

\end{minipage}%
\hfill
\begin{minipage}[r]{0.5\textwidth}
\begin{quotation}
\begin{doublespace}

\textit{À mes parents et tous ceux qui ont été patient...}


\end{doublespace}
\end{quotation}
\end{minipage}%

\cleardoublepage

% ----------------------------------------------------------------------%


% ----------------------------------------------------------------------%
% 2- Remerciements.                                                    %
% ----------------------------------------------------------------------%

\remerciements

\selectlanguage{frenchb}

Je tiens dans un premier temps à remercier mon directeur Dominique Gravel pour m'avoir donné l'opportunité de réaliser cette maitrise. Je tiens également à remercier Isabelle Boulangeat et Matthew Talluto pour leurs implications et .

% ----------------------------------------------------------------------%
% 3- Avant-propos.                                                     %
% ----------------------------------------------------------------------%

\avantpropos

\selectlanguage{frenchb}

Je pourrais commencer cette avant-propos par, \textit{Depuis tout jeune, je souhaite comprendre
comment la nature fonctionne...} mais je ne le ferais pas. Je n'ai pas atteri sur ce projet par
simple mouvement brownien. Cette maitrise est plutôt l'aboutissement d'un parcours scolaire plus ou
moins linéaire animé par ma curiosité et ma passion pour les sciences environnementales. Après une
technique en Bioécologie et un baccalauréat en Biologie, je me suis apercu que mes centres
d'interêts ne portaient pas sur une espèce en particulier. J'étais davantage fasciné par les
lombrics, que par l'orignal, de par l'important rôle fonctionelle qu'ils remplissent. Ainsi, je ne
souhaitais pas devenir un écologiste spécialiste mais plutôt un écologiste généraliste attiré par le
désir de comprendre comment un écosystème fonctionne. Mon implication au sein du laboratoire de
Dominique Gravel, durant mon baccalauréat, m'a permise d'être initier à l'univers de la modélisation
et de l'écologie théorique. Ce domaine offrait le cadre approprié pour réalisé ma maitrise. Mon
cheminement n'est donc pas animé par un simple mouvement brownien, mais plutot par une passion, une
réfléxion, des rencontres et une opportunité qui ont permis de déterminer qui je suis aujourd'hui.

Dimension éco-informatique,

Allez retournons 
La repoductibitité est un des critères scientifiques (Popper)

[Cette page est facultative; l’éliminer si elle n’est pas utilisée. L’avant-propos ne doit pas être confondu avec l'introduction. Il n’est pas d’ordre scientifique alors que l’introduction l’est. Il s’agit d'un discours préliminaire qui permet notamment à l'auteur d'exposer les raisons qui l'ont amené à étudier le sujet choisi, le but qu'il veut atteindre, ainsi que les possibilités et les limites de son travail. On peut inclure les remerciements à la fin de ce texte au lieu de les présenter sur une page distincte.]

% ----------------------------------------------------------------------%
% 4- Resume/Abstract                                                           %
% ----------------------------------------------------------------------%

\resume
\begin{singlespace}

  De nombreuses espèces ne migrent pas assez vite pour suivre la rapidité des changements
  climatiques. Les arbres sont bien connus pour éprouver de longs délais dans leurs réponses au
  climat parce qu'ils sont sessiles, possèdent une forte longévité et disposent de faible capacité
  de dispersion. Les approches actuelles pour prédire l'aire de répartition future des espèces,
  telles que les modèles d'enveloppe climatique, ne peuvent pas tenir compte de ces particularités
  propres aux écosystèmes forestiers, car ils assument une dispersion infinie et une réponse
  instantanée aux changements climatiques. Nous proposons une nouvelle approche de modélisation
  basée sur la théorie des métapopulations pour tenir compte de cette capacité limitée de
  dispersion, des interactions biotiques et de la démographie propre à la forêt tempérée nordique du
  nord-est de l'Amérique du Nord. Notre objectif est d'évaluer si ce biome forestier sera en mesure
  de suivre sa niche climatique d'ici la fin de ce siècle. Nous avons effectué des simulations de
  l'écotone entre la forêt boréale et tempérée en utilisant un modèle d'états et de transitions
  (STM), dans lequel les communautés forestières sont classées dans 4 états: boréales, tempérées,
  mélangées et en régénération après une perturbation. Les transitions entre les états sont
  calibrées à partir des inventaires des parcelles permanentes présents aux États-Unis et au Canada.
  Les résultats des simulations du modèle indiquent que la forêt tempérée se déplacera seulement de
  14 $\pm$ 2,0 km alors qu'un modèle de distribution d'espèces standard prédit un déplacement de
  238,79 $\pm$ 34,24 km. Les simulations de l'écotone forestier mettent également en évidence que la
  majorité des transitions attendues seront une conversion des peuplements mixtes vers des
  peuplements purement décidus. L'utilisation du modèle avec un scénario de dispersion infinie
  révèle que les interactions biotiques et la démographie sont les facteurs les plus importants qui
  limitent la capacité d'expansion du biome de la forêt tempérée. En conclusion, la forêt tempérée
  possède une faible résilience au changement climatique en raison de sa lente démographie et des
  fortes interactions compétitives avec les espèces boréales.


  \begin{quote}
    Mots clés: [Inscrire ici 5 à 10 mots clés]
  \end{quote}
\end{singlespace}
\cleardoublepage

\abstract
\begin{singlespace}

  Many species are not migrating fast enough to keep pace with the rapidly changing climate. Trees are well known to experience long time lags in their migration responses because they are sessile, long-lived and have a relatively short dispersal ability. Actual approaches to forecast range shifts under climate change, such as Species Distribution Models, cannot account for the particularities of forest ecosystems because they assume infinite dispersal and instantaneous response to climate change. Here, we propose a new modelling approach based on metapopulation theory to account for dispersal limitations, biotic interactions and the demography of the temperate forest. Our objective is to assess if the North-Eastern American temperate forest will be able to track its climatic optimum by the end of this century. Transitions among states are calibrated from several long-term forest plots surveys from United States and Canada. We find that even if standard species distribution models would predict a northward shift of the temperate forest distribution of 328$\pm$28.4 km, the temperate forest will barely move 14$\pm$2.0 km into the boreal forest at the end of this century. We also find than most of the expected transitions will be the conversion from mixed to pure temperate stands. A comparison with an infinite dispersal scenario reveals that biotic interactions and stand replacement dynamics are the most significant factors limiting migration rate of forest trees. We conclude that the temperate forest has a low resilience to climate change because of their low demography and competitive interactions with resident trees.
  
  \begin{quote}
    Keywords: [Inscrire ici 5 à 10 mots clés]
  \end{quote}
\end{singlespace}
\cleardoublepage

% ----------------------------------------------------------------------%
% 5- Table des matières.                                               %
% ----------------------------------------------------------------------%

\tabledesmatieres

% ----------------------------------------------------------------------%
% 6- Liste des tableaux.                                               %
% ----------------------------------------------------------------------%

\listedestableaux

% ----------------------------------------------------------------------%
% 7- Table des matières.                                               %
% ----------------------------------------------------------------------%

\listedesfigures


% ----------------------------------------------------------------------%
% Fin des liminaires.                                                  %
% ----------------------------------------------------------------------%

\cleardoublepage

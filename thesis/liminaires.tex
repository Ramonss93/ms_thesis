% Liminaires de la thèse.                                              %
% UQAR septembre 2013                                                  %
% ---------------------------------------------------------------------%

% ----------------------------------------------------------------------%
% 1- Page titre.                                                        %
% ----------------------------------------------------------------------%

\Pagetitre
\cleardoublepage
% ----------------------------------------------------------------------%
% inclusions qui pourraient mériter d'être incluses dans le .cls
% (commentez si non-nécessaire)
% 1.1 - Composition du Jury.                                           %
\thispagestyle{empty}

\null
\vfill
\noindent \textbf{Composition du jury:}\\
\vspace{1cm}

\begin{singlespace}
  \noindent \textbf{Dominique Berteaux, président du jury, Université du Québec à Rimouski}\\

  \noindent \textbf{Dominique Gravel, directeur de recherche, Université du Québec à Rimouski}\\

  \noindent \textbf{Matthew Talluto, codirecteur de recherche, Université Joseph Fourier}\\

  \noindent \textbf{Isabelle Boulangeat, codirecteur de recherche, Aarhus University}\\

  \noindent \textbf{Niklaus Zimmermann, examinateur externe, Swiss Federal Research Institute WSL}\\
\end{singlespace}

\vspace{2cm}
\noindent Dépôt initial le [date mois année]
\hspace{3cm}
Dépôt final le [date mois année]


\cleardoublepage

% % 1.2 - Avertissement biblio.
\thispagestyle{empty}

\vspace{2cm}
\begin{center}
UNIVERSITÉ DU QUÉBEC À RIMOUSKI\\
Service de la bibliothèque
\end{center}

\vspace{3cm}
\begin{center}
Avertissement
\end{center}


\vspace{1cm}

\noindent La diffusion de ce mémoire ou de cette thèse se fait dans le respect des droits de son auteur, qui a signé le formulaire {\itshape \og Autorisation de reproduire et de diffuser un rapport, un mémoire ou une thèse \fg}. 
En signant ce formulaire, l’auteur concède à l’Université du Québec à Rimouski une licence non exclusive d’utilisation et de publication de la totalité ou d’une partie importante de son travail de recherche pour des fins pédagogiques et non commerciales. 
Plus précisément, l’auteur autorise l’Université du Québec à Rimouski à reproduire, diffuser, prêter, distribuer ou vendre des copies de son travail de recherche à des fins non commerciales sur quelque support que ce soit, y compris l’Internet. 
Cette licence et cette autorisation n’entraînent pas une renonciation de la part de l’auteur à ses droits moraux ni à ses droits de propriété intellectuelle. 
Sauf entente contraire, l’auteur conserve la liberté de diffuser et de commercialiser ou non ce travail dont il possède un exemplaire.



\cleardoublepage
% % 1.3 - Dedicace.
\thispagestyle{empty}

\begin{minipage}[l]{0.45\textwidth}

\end{minipage}%
\hfill
\begin{minipage}[r]{0.5\textwidth}
\begin{quotation}
\begin{doublespace}

\textit{À mes parents et tous ceux qui ont cru en moi ...}


\end{doublespace}
\end{quotation}
\end{minipage}%

\cleardoublepage

% ----------------------------------------------------------------------%


% ----------------------------------------------------------------------%
% 2- Remerciements.                                                    %
% ----------------------------------------------------------------------%

\remerciements

\selectlanguage{frenchb}

Je tiens dans un premier temps à remercier mon directeur Dominique Gravel pour m'avoir donné
l'opportunité de réaliser cette maîtrise. Dominique m'a redonné confiance en moi et a su me redonner
espoir dans mes capacités en m'accompagnant à travers les étapes clés de ce mémoire. Il m'a impressionné par sa patience et ses nombreuses qualités humaines et m'a montré
que la science est avant tout un travail d'équipe.

Je remercie également mes deux co-superviseurs, Isabelle Boulangeat et Matthew Talluto, pour leur
travail, leur implication et leur dévouement dans la réalisation de ce projet. Isabelle et Matt sont
des personnes avec qui j'ai grandement apprécié travailler et desquels j'ai beaucoup appris. J'espère continuer à collaborer avec eux sur d'autres projets futurs.

Merci à Nicolas Casajus, Kevin Cazelles et Philippe Desjardins Proulx pour leurs conseils et aides
techniques sur R et C qui ont été d'une grande valeur. Merci à Timothée Poisot pour la confiance et
le soutien dont il a su me témoigner au début du projet. Un merci particulier à James Caveen pour sa
disponibilité et son support. Il a su me communiquer sa passion et m'a fourni les bons outils pour
mener à bien ce projet.

%Merci également à Marie-Josée Naud et Pierre Legagneux pour votre écoute

Lors de ces deux (+0.75) années de maîtrise, j'ai également eu des partenaires de science (et de
soirées) que je me dois d'immortaliser sur ce mémoire. Je remercie Claire Jacquet, Amaël Lesquin,
Renaud McKinnon, Idaline Laigle, Kevin Solarik, Kevin Cazelles, Camille Albouy, David Beauchesne,
Hedvig Nenzén et Jonathan Brassard pour leurs complicités et pour avoir fait en sorte que ce
laboratoire ne soit pas juste une coquille vide. J'ai grandi personnellement et professionnellement
grâce à plusieurs d'entre vous.

Je tiens à remercier ma blonde pour sa patience exemplaire et pour m'avoir pardonné mes moments
d'absence devant mon bol de céréales le matin. Enfin, je dédie ce mémoire à ma famille qui même de
loin a su me témoigner un soutien indispensable pour tourner cette page.

% ----------------------------------------------------------------------%
% 3- Avant-propos.                                                     %
% ----------------------------------------------------------------------%

\avantpropos

\selectlanguage{frenchb}

Je pourrais commencer cet avant-propos par, \textit{"Depuis tout jeune, je souhaite comprendre
comment la nature fonctionne..."}, mais je ne le ferai pas. Je n'ai pas abouti sur ce projet par
simple mouvement brownien. Après une technique en Bioécologie et un baccalauréat en Biologie, je me
suis aperçu que mes centres d'intérêt ne portaient pas sur une espèce en particulier, mais plutôt
sur la diversité des organismes et les rôles fonctionnels qu'ils remplissent au sein d'un
écosystème. J'étais autant fasciné par les espèces du genre \textit{Lombricus sp.} que par
l'orignal. Ma curiosité me poussait donc vers la compréhension des mécanismes écologiques à une plus
grande échelle. Mon implication au sein du laboratoire de Dominique Gravel durant mon baccalauréat,
m'a permis d'être initié à l'écologie théorique et la biogéographie ainsi qu'à l'approche
scientifique par modélisation. Ce laboratoire m'offrait donc un cadre idéal pour réaliser ma
maîtrise et satisfaire cette curiosité.

\noindent\textbf{Reproductibilité de mes travaux}

Le père du modèle hypothético-déductif, Karl Popper, énonçait que l'un des critères fondamentaux
dans la réalisation d'une bonne étude scientifique réside dans la reproductibilité de la méthode.
Pour atteindre ce critère, des outils informatiques permettent aujourd'hui de dépasser la simple
description méthodologique sur papier. Dans ce contexte, l'ensemble de mes scripts et programmes
utilisé pour ce projet sont disponibles librement via une plateforme internet
(système de contrôle de version). Ainsi, le code nécessaire à la conceptualisation de la base de
données QUICC-FOR est accessible à travers le dépôt: \url{https://github.com/QUICC-FOR/QUICCSQL}. Ce
dépôt ne contient pas les données biologiques (celles-ci demeurant la propriété des Ministères ou
entreprises privées partenaires du projet) mais seulement l'information permettant de reproduire
l'architecture de la base de données SQL. Les scripts nécessaires à l'extraction des données pour la
calibration et les projections du modèle sont accessibles à cette adresse
\url{https://github.com/QUICC-FOR/STModel-Data}. Ils permettent de retracer l'ensemble des filtres
et des étapes de manipulation des données nécessaires aux analyses. Le modèle
d'automate cellulaire utilisé (States and Transitions model, STM) pour les simulations est également
disponible (\url{https://github.com/QUICC-FOR/STModel-Simulation}), ainsi que les étapes de
calibration (\url{https://github.com/QUICC-FOR/STModel-Calibration}) pour l'obtention des paramètres
par la méthode de maximum de vraisemblance. Enfin, le post-traitement des simulations et les codes
nécessaires à la production des figures sont disponibles à cette adresse:
\url{https://github.com/QUICC-FOR/STModel-CompAnalysis}.

La mise à disposition de ces ressources constitue un gage de transparence auprès de mes pairs. Elle
me permet également de valoriser mes compétences professionnelles en programmation scientifique.
Enfin, elle garantit la possibilité de conduire les mêmes analyses dans les 20 ou 30 prochaines
années lorsque de nouvelles observations/données seront accessibles; un critère indéniable lorsque
l'on connaît la lenteur à laquelle un écosystème forestier se réajuste aux changements
environnementaux.


% ----------------------------------------------------------------------%
% 4- Resume/Abstract                                                           %
% ----------------------------------------------------------------------%

\resume
\begin{singlespace}

  De nombreuses espèces ne migrent pas assez vite pour suivre la rapidité des changements climatiques. Les arbres sont bien connus pour éprouver de longs délais dans leurs réponses au climat parce qu'ils sont sessiles, qu'ils possèdent une forte longévité et qu'ils disposent d'une faible capacité de dispersion. Les approches actuelles pour prédire l'aire de répartition future des espèces, telles que les modèles d'enveloppe climatique, n'intègrent pas ces particularités propres aux écosystèmes forestiers, car elles assument une dispersion infinie et une réponse instantanée aux changements climatiques. Nous proposons une nouvelle approche de modélisation basée sur la théorie des métapopulations pour tenir compte de cette capacité limitée de dispersion, des interactions biotiques et de la démographie propre à la forêt tempérée nordique du nord-est de l'Amérique du Nord. Notre objectif est d'évaluer si ce biome forestier sera en mesure de suivre sa niche climatique d'ici la fin de ce siècle. Nous avons effectué des simulations de l'écotone entre la forêt boréale et la forêt tempérée en utilisant un modèle d'états et de transitions (STM), dans lequel les communautés forestières sont classées dans 4 états: boréales, tempérées, mélangées et en régénération après une perturbation. Les transitions entre les états sont calibrées à partir des inventaires des parcelles permanentes présents aux États-Unis et au Canada. Les résultats des simulations du modèle indiquent que la forêt tempérée se déplacera seulement de 14 $\pm$ 2,0 km alors qu'un modèle de distribution d'espèces standard prédit un déplacement de 238,79 $\pm$ 34,24 km. Les simulations de l'écotone forestier mettent également en évidence que la majorité des transitions attendues seront une conversion des peuplements mélangées vers des peuplements purement décidus. L'utilisation du modèle avec un scénario de dispersion infinie révèle que les interactions biotiques et la démographie sont les facteurs les plus importants limitant la capacité d'expansion du biome de la forêt tempérée. En conclusion, la forêt tempérée possède une faible résilience aux changements climatiques en raison de sa lente démographie et des fortes interactions compétitives avec les espèces boréales.

  \begin{quote}
    Mots clés: Biogeographie, changement climatique, dispersion, arbres, dynamique de régénération.
  \end{quote}
\end{singlespace}
\cleardoublepage

\abstract
\begin{singlespace}

  Many species are not migrating fast enough to keep pace with the rapidly changing climate. Trees are well known for experiencing long time lags in their migration responses because they are sessile, long-lived and have relatively short dispersal abilities. Actual approaches to forecast range shifts under climate change, such as Species Distribution Models, cannot account for the particularities of forest ecosystems because they assume infinite dispersal and instantaneous response to climate change. Here, we propose a new modelling approach based on metapopulation theory to account for dispersal limitations, biotic interactions and the demography of the temperate forest. Our objective is to assess if the North-Eastern American temperate forest will be able to track its climatic optimum by the end of this century. Transitions among states are calibrated from several long-term forest plots surveys from United States and Canada. We find that even if standard species distribution models would predict a northward shift of the temperate forest distribution of 328$\pm$28.4 km, the temperate forest will barely move 14$\pm$2.0 km into the boreal forest at the end of this century. We also find than most of the expected transitions will be the conversion from mixed to pure temperate stands. A comparison with an infinite dispersal scenario reveals that biotic interactions and stand replacement dynamics are the most significant factors limiting migration rate of forest trees. We conclude that the temperate forest has a low resilience to climate change because of their low demography and competitive interactions with resident trees.
  \begin{quote}
    Keywords: Biogeography, climate change, dispersion, trees, stand replacement dynamics.
  \end{quote}
\end{singlespace}
\cleardoublepage

% ----------------------------------------------------------------------%
% 5- Table des matières.                                               %
% ----------------------------------------------------------------------%

\tabledesmatieres

% ----------------------------------------------------------------------%
% 6- Liste des tableaux.                                               %
% ----------------------------------------------------------------------%

\listedestableaux

% ----------------------------------------------------------------------%
% 7- Table des matières.                                               %
% ----------------------------------------------------------------------%

\listedesfigures


% ----------------------------------------------------------------------%
% Fin des liminaires.                                                  %
% ----------------------------------------------------------------------%

\cleardoublepage

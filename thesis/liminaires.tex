%----------------------------------------------------------------------%
% Liminaires de la thèse.                                              %
% UQAR septembre 2013                                                  %
% ---------------------------------------------------------------------%

% ----------------------------------------------------------------------%
% 1- Page titre.                                                        %
% ----------------------------------------------------------------------%

\Pagetitre
\cleardoublepage
% ----------------------------------------------------------------------%
% inclusions qui pourraient mériter d'être incluses dans le .cls
% (commentez si non-nécessaire)
% 1.1 - Composition du Jury.                                           %
\thispagestyle{empty}

\null
\vfill
\noindent \textbf{Composition du jury:}\\
\vspace{1cm}

\begin{singlespace}
  \noindent \textbf{Dominique Arsenault, président du jury, Université du Québec à Rimouski}\\

  \noindent \textbf{Dominique Gravel, directeur de recherche, Université du Québec à Rimouski}\\

  \noindent \textbf{Matthew Talluto, codirecteur de recherche, Université Joseph Fourier}\\

  \noindent \textbf{Isabelle Boulangeat, codirecteur de recherche, Université du Québec à Rimouski}\\

  \noindent \textbf{[Prénom Nom], examinateur externe, [Université d’attache]}\\
\end{singlespace}

\vspace{2cm}
\noindent Dépôt initial le [date mois année]
\hspace{3cm}
Dépôt final le [date mois année]


\cleardoublepage

% % 1.2 - Avertissement biblio.
\input{avertissement.tex}
% % 1.3 - Dedicace.
\thispagestyle{empty}

\begin{minipage}[l]{0.45\textwidth}

\end{minipage}%
\hfill
\begin{minipage}[r]{0.5\textwidth}
\begin{quotation}
\begin{doublespace}

\textit{À mes parents et tous ceux qui ont été patient...}


\end{doublespace}
\end{quotation}
\end{minipage}%

\cleardoublepage

% ----------------------------------------------------------------------%


% ----------------------------------------------------------------------%
% 2- Remerciements.                                                    %
% ----------------------------------------------------------------------%

\remerciements

\selectlanguage{frenchb}

Je tiens dans un premier temps à remercier mon directeur Dominique Gravel pour m'avoir donné l'opportunité de réaliser cette maitrise. Je tiens également à remercier Isabelle Boulangeat et Matthew Talluto pour leurs implications et .

% ----------------------------------------------------------------------%
% 3- Avant-propos.                                                     %
% ----------------------------------------------------------------------%

\avantpropos

\selectlanguage{frenchb}

[Cette page est facultative; l’éliminer si elle n’est pas utilisée. L’avant-propos ne doit pas être confondu avec l'introduction. Il n’est pas d’ordre scientifique alors que l’introduction l’est. Il s’agit d'un discours préliminaire qui permet notamment à l'auteur d'exposer les raisons qui l'ont amené à étudier le sujet choisi, le but qu'il veut atteindre, ainsi que les possibilités et les limites de son travail. On peut inclure les remerciements à la fin de ce texte au lieu de les présenter sur une page distincte.]

% ----------------------------------------------------------------------%
% 4- Resume/Abstract                                                           %
% ----------------------------------------------------------------------%

\resume
\begin{singlespace}

  [Le résumé en français doit présenter en 350 mots maximum pour un mémoire et en 700 mots pour une thèse : (1) le but de la recherche, (2) les sujets étudiés, (3) les hypothèses de travail et la méthode utilisée, (4) les principaux résultats et (5) les conclusions de l'étude ou de la recherche.]

  \begin{quote}
    Mots clés: [Inscrire ici 5 à 10 mots clés]
  \end{quote}
\end{singlespace}
\cleardoublepage

\abstract
\begin{singlespace}

  [L'abstract doit être une traduction anglaise fidèle et grammaticalement correcte du résumé en français.]

  \begin{quote}
    Keywords: [Inscrire ici 5 à 10 mots clés]
  \end{quote}
\end{singlespace}
\cleardoublepage

% ----------------------------------------------------------------------%
% 5- Table des matières.                                               %
% ----------------------------------------------------------------------%

\tabledesmatieres

% ----------------------------------------------------------------------%
% 6- Liste des tableaux.                                               %
% ----------------------------------------------------------------------%

\listedestableaux

% ----------------------------------------------------------------------%
% 7- Table des matières.                                               %
% ----------------------------------------------------------------------%

\listedesfigures


% ----------------------------------------------------------------------%
% Fin des liminaires.                                                  %
% ----------------------------------------------------------------------%

\cleardoublepage

\documentclass[10pt]{article}
\usepackage[margin=1in]{geometry}

% use proper unicode fonts
\usepackage[T1]{fontenc}
\usepackage[utf8]{inputenc}

\usepackage{amsmath} % for better display of equations
\usepackage{amssymb}
\usepackage{amsthm}

\usepackage{setspace}
\usepackage[square,sort,comma,numbers]{natbib}
\bibliographystyle{siam}

\onehalfspace

%% Typographic adjustement
\usepackage{microtype}
\usepackage{lipsum}

\usepackage{caption}
\usepackage{subcaption}
\captionsetup{labelfont={bf,small}, textfont={small}}

\setlength{\columnsep}{1cm}

\usepackage{titling} % controls the way the title information is displayed
\pretitle{\begin{flushleft}\Large}
\posttitle{\end{flushleft}}
\predate{}
\postdate{}
\preauthor{\begin{flushleft}}
\postauthor{\end{flushleft}}
\setlength{\droptitle}{-3em}

\setlength{\parskip}{0.5em}

\usepackage{authblk} % adds some nice options for displaying the author list
\renewcommand\Authsep{\protect\\}
\renewcommand\Authands{\protect\\}

%% graphics packages
\usepackage{graphicx}
%\usepackage[nomarkers, tablesfirst]{endfloat} % for final
%\captionsetup{labelsep=none,textformat=empty} % for final
\captionsetup{labelformat=simple} % for drafts
\usepackage{booktabs}
\usepackage[flushleft]{threeparttable}
\usepackage{geometry}

\usepackage{tikz}
\usepackage{pgfplots}
\usepackage{pgfplotstable}
\usepackage{xcolor}
\usepackage{graphicx}
\usetikzlibrary{shapes, arrows, positioning}

\usepackage[bitstream-charter]{mathdesign}

% State colors
\definecolor{Temperate}{RGB}{254,172,24}
\definecolor{Boreal}{RGB}{24,148,148}
\definecolor{Regeneration}{RGB}{252,101,69}
\definecolor{Mixed}{RGB}{135,209,134}

%% ----------------------------------
%
%     Title and authorship information
%
%% ----------------------------------


\title{Slow demography constrains the northward expansion of the temperate forest under climate change}

% Title 2
% \title{\textbf{What constrains the northward expansion of the temperate forest?}}

\date{}
\author[1,*]{Steve Vissault (s.vissault@yahoo.fr)}
\author[1,2]{Matthew V. Talluto (mtalluto@gmail.com)}
\author[1]{Isabelle Boulangeat (isabelle.boulangeat@gmail.com)}
\author[1]{Dominique Gravel (dominique\_gravel@uqar.ca)}
\affil[1]{Département de Biologie, Géographie et Chimie, Université du Québec à Rimouski, Rimouski, Québec, Canada}
\affil[2]{Laboratoire d'Écologie Alpine, Université Joseph Fourier, Grenoble, France}
\affil[*]{Author for correspondance. Address: Departement de Biologie, Chimie, et Géographie, 300, Allée des Ursulines, Rimouski, Quebec G5L 3A1, Canada}

\begin{document}

\begin{titlingpage}
		\maketitle

		\begin{flushleft}

			\textbf{Short title:} Simulate the eastern boreal-temperate forests ecotone under climate change.

			\textbf{Keywords:} states and transitions model, patch occupancy, landscape dynamics, forest inventory databases, community range shift.
		\end{flushleft}
	\end{titlingpage}
	%TC:endignore
	%TC:break abstract


\begin{abstract}
	\noindent
	\textit{\lipsum[1]}
\end{abstract}

\section*{Introduction}

Most studies use correlative approaches to predict impact of CC on future distribution range shift leading to instantaneous vegetation responses.

Firstly, we investigate the migration rate of the temperate forest using a states and transitions model.  Secondly, we used different versions of the state and transition model to investigate which ecological processes such as the dispersion, or either demography is limiting or increasing the migration ability of the temperate forest.

The demography, the limited dispersion of temperate species and the spatial dynamics are delaying the temperate forest migration.

J'ai besoin de définir ce que c'est que la dynamique spatiale de la forêt tempérée.

\section*{Methods}

The temperate-boreal forests ecotone can be seen at the landscape scale as a macro-mosaic filled by three different forest stand patches; Boreal stand dominated by coniferous species, Temperate stand dominated by broadleaf species and finally Mixed stand as a mid-succesionnal patch. In the first section, we present how we extracted and classified forest plots surveys into those three regional forest biomes and linked the plots locations to climatic data. In the second section, we described the model allowing us to simulate the dynamic of the boreal-temperate forest ecotone and then focused on the model calibration. In the last section, we explained the simulation plan and the different model versions we ran to assess which ecological mechanisms will constrained the migration rate of the temperate forest.

\subsection*{Plot surveys into community patches}

\subsubsection*{Forest Inventory databases}

We used 4 forest inventory databases widely distributed in Eastern North America, from West-Virginia (US) to Quebec (CAN) (include plots distribution and study area in figs). We selected N plot surveys located at the boreal temperate ecotone (\textit{add coordinates of the study area}) and then classify each plot measurement in the four states following the species composition. \textit{add description on measurements}. A forest state is defined as mature stand characterized by a specific species community which is the result of the local climatic conditions.
% Ajouter les implications d'une telle forme de classification. De la pure écologie forestière vue par Cléments.

 In our case, the temperate community consists of 8 different species (\textit{full species list}) and the boreal community 7 species (\textit{ species list}). If one those species is present in the stand, then the patch is classified as a mixed state.

We filtered out all trees with diameter at breast height lesser than 12,7 cm.

\subsubsection*{Climatic database}

For each plot location, we extract the climate of the last 15 years previous the year of measurement.


\subsection*{The states and transitions model approach}

\subsubsection*{States description}

To reproduce and simulate the dynamic of this ecotone, we used a state and transition model as a patch occupancy model \cite{Leibold2004} including the three different forest stand types as states: Boreal (B), Temperate (T), Mixed (M) (Fig. \ref{fig1}).  The disturbance regime is one of the important component of this natural system dynamic \cite{Vanderwel2014}; consequently, we added the regeneration state (R, Fig. \ref{fig1}) to represent a post-disturbance stand characterized by early successional species and a low basal area. Among the states, transitions occur by ecological processes. For instance, a temperate patch is converted as a mixed patch by colonization of boreal species. Then this mixed stand can transform to a pure temperate patch by competitive exclusion of boreal species. When a disturbance appears on the patch such as fire, wind throw or insect outbreak, the patch is transferred has a regeneration state. The disturbed patch can recover from this disturbance to a boreal, temperate or mixed stand by successional dynamic.

\begin{figure}
\begin{center}
	\tikzstyle{Temperate}=[circle,
	    thick,
	    minimum size = 1.2cm,
	    inner sep =5pt,
	    draw=Temperate,
	    fill=Temperate]
	\tikzstyle{Boreal}=[circle,
	    thick,
	    minimum size = 1.2cm,
	    inner sep =5pt,
	    draw=Boreal,
	    fill=Boreal]
	\tikzstyle{Regeneration}=[circle,
	    thick,
	    minimum size = 1.2cm,
	    inner sep =5pt,
	    draw=Regeneration,
	    fill=Regeneration]
	\tikzstyle{Mixed}=[circle,
	    thick,
	    minimum size = 1.2cm,
	    inner sep =5pt,
	    draw=Mixed,
	    fill=Mixed]

	\begin{tikzpicture}[->,>=stealth',auto,scale=0.60]
		\node [circle,Mixed] (M) at (0,0) {M};
		\node [circle,Boreal] (B) at (-8,5) {B};
		\node [circle,Temperate] (T) at (8,5) {T};
		\node [circle,Regeneration] (R) at (0,10) {R};

		\path	(M) edge [thick,loop below,-latex]  node {} (M);
		\path	(T) edge [thick,loop right,-latex]  node {} (T);
		\path	(B) edge [thick,loop left,-latex]  node {} (B);
		\path	(R) edge [thick,loop above,-latex]  node {} (R);

		\draw[thick,-latex] (M) to node[above,sloped] {$\theta (1-\theta_T)(1-\epsilon)$} (B);
		\draw[thick,-latex] (B) to[bend right=25] node[below,sloped] {$\beta_T  (T+M)(1-\epsilon) $} (M);

		\draw[thick,-latex] (T) to[bend left=25] node[below,sloped] {$\beta_B(B+M)(1-\epsilon)$} (M);
		\draw[thick,-latex] (M) to node[above,sloped] {$\theta \cdot \theta_T (1-\epsilon)$} (T);

		\draw[thick,-latex] (R) to[bend left=25] node[above,sloped] {$\alpha_T(T+M)[1-\alpha_B(B+M)]$} (T);
		\draw[thick,-latex] (T) to node[below,sloped] {$\epsilon$} (R);

		\draw[thick,-latex] (R) to[bend right=25] node[above,sloped] {$\alpha_B(B+M)[1-\alpha_T(T+M)]$} (B);
		\draw[thick,-latex] (B) to node[below,sloped] {$\epsilon$} (R);

		\draw[thick,-latex,transform canvas={xshift=0.8ex}] (R) to node[above,sloped] {$\alpha_B(M + B) \cdot \alpha_T(M + T)$} (M);
		\draw[thick,-latex,transform canvas={xshift=-0.8ex}] (M) to node[above,sloped] {$\epsilon$} (R);
	\end{tikzpicture}
\end{center}

\caption{The states and transitions model illustrating all states and possible transition in the boreal-temperate forest system. B, T, M and R respectively mean; Boreal, Temperate, Mixed and Regeneration. Each arrow represent a transition between state.}
\label{fig1}

\end{figure}


Each of these ecological mechanisms are formulated in the present model as a transition probability. All Transitions between states are possible except the direct transition between a temperate and boreal stand, which requires an intermediate step through the state mixed (\textit{Need to give an example or cite something ?}). Transition probabilities between states are climate-dependant based on two climatic variables: annual precipitation (mm) and annual mean temperature ($^{\circ}$C). Except for the colonization probability, transition probabilities vary with the proportion of coniferous or deciduous found in the neighborhood to represent propagule pressure.  Hence, the transition of a boreal patch towards a mixed patch depends on local environmental condition and on the availability of temperate seeds present in mixed and temperate patches surrounding the temperate patch.


% # Filters:
% 	# - DBH > 127 and not null
% 	# - plot_size is not null
% 	# - plot_id should be in the materialized view (for further details see ./src_sql/stm_plot_ids.sql)
% 	# - tree is not dead

Then we computed a transition matrix within N states transition occurred between two measurements.

\begin{table}
	\begin{center}
		\caption{Transition and none-transition (diagonal) observed between two measurements through all plots surveys extract from databases.}
		\label{TransMat}
		\begin{tabular}{c|cccc}
			&	\textbf{B} &     \textbf{M} &     \textbf{R} &     \textbf{T} \\
			\hline
			\textbf{B} & \textbf{15 358} &   794 &   203 &     0 \\
			\textbf{M} &   302 & \textbf{14 433} &    51 &   960 \\
			\textbf{R} &   485 &    57 &   \textbf{209} &    80 \\
			\textbf{T} &     0 &   891 &    40 & \textbf{15 216}
		\end{tabular}
	\end{center}
\end{table}

%Supplementary materials: US data are provided by the Forest Inventory and Analysis National Program and included 86.0000 forest plots standardized since 1990 and monitored until 2013 with up to 4 measurements by forest plot. Quebec data are provided from the Ministère des Forêts de la Faune et des Parcs with 12.409 permanent plots and DOMTAR, a forest company in paper production with 1.741 plots. Quebec plots surveys started in 1960 until 2011 with up to 10 measurements. Ontario and New-Brunswick included 1.038 and 2.748 plots respectively. Ontario monitored forest plots since 1992 until 2006 with up to 3 measurement, and New-Brunswick since 1985 to 2010 with up to 7 measurements.

\subsubsection*{Transition probabilities}

We estimated each transition probabilities using logistic regression with the annual mean temperature and the annual precipitation as independent variables.

% Comment les infos sur le voisinage sont intégré dans le ft ?

The seven model parameters were fitted simultaneously using simulating annealing (GenSA package \cite{YangXiang2013}).

We used all transitions observed and selected the second degree polynomial as the best model fitted.

\begin{equation}
	logit(\alpha_b) = \alpha_{b0} + \alpha_{b1} \cdot TP + \alpha_{b2} \cdot PP + \alpha_{b3} \cdot TP^2 + \alpha_{b4} \cdot PP^2
\end{equation}

The combination of temperature and precipitation are the best set of climatic variables to explain the distribution of the boreal forest \cite{Scheffer2012}.



To calibrate the transition probabilities between states based on climate and plot neighbors, we used R (version 3.2.0) and the classification algorithm Random Forest (RandomForest package, version 4.6-10 ) \cite{Liaw2002a}, we incorporated three information types: (1) state transitions observed between plot measurements, (2) the average climate of the 15 years before each measurements for the two climatic variables of interest and finally the (3) the proportion of states available in the neighbors using a SDM (RandomForest) approach as a proxy.

% Ajouter une figure avec le workflow de la calibration
\subsection*{Simulations and analysis}



% Ajouter la stats du HK pour la validation croisé du modèle.

% TODO
% - Ajouter les params sur la figure du modèle
% - Ajouter les function quadratiqes


\clearpage
\bibliography{/Users/steve/Documents/BibteX/Master}

\end{document}
